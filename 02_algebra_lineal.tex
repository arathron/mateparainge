%%%%%%%%%%%%%%%%%%%%%%%%%%%%%%%%%%%%%%%%%%%%%%%%%%%%%%%%%%%%%%%%%%%%%%%%%%%%%%%
% Álgebra lineal
%
% Copyright (c) 2016 Damián Silvestre. Permission is granted to copy, 
% distribute and/or modify this document under the terms of the 
% GNU Free Documentation License, Version 1.3 or any later version published by
% the Free Software Foundation; with no Invariant Sections, no 
% Front-Cover Texts, and no Back-Cover Texts. 
%
% Details of the GNU FDL can be found here: 
% http://www.gnu.org/licenses/licenses.html
%
%%%%%%%%%%%%%%%%%%%%%%%%%%%%%%%%%%%%%%%%%%%%%%%%%%%%%%%%%%%%%%%%%%%%%%%%%%%%%%%

\part{Álgebra lineal}

\chapter{Espacios Vectoriales}


\section{Estructuras algebraicas}

\begin{definition}
Un \textbf{grupo} es un conjunto $G$ con una función $\cdot : G \times G \to G$ tal que

\begin{enumerate}
\item Es asociativa, para todo $a,b,c \in G$ se tiene
$(a \cdot b) \cdot c = a \cdot (b \cdot g)$

\item Existe elemento neutro $ e \in G$ tal que
$ e \cdot g = g \cdot e = g$ (para todo $ g \in G$)

\item Existe elemento inverso $ g^{-1} \in G$ para cada $ g \in G$ de forma tal que $ g \cdot g^{-1} = g^{-1} \cdot g = e$

\end{enumerate}

Si además cumple ser \textbf{conmutativa}, es decir que para todo $ a,b \in G$ se tiene $ a \cdot b = b \cdot a$, se dice que es un \textbf{grupo abeliano} o conmutativo y se suele usar la notación aditiva ($+$ para la función, $-a$ para el inverso).

Un \textbf{monoide}, es como la definición de grupo, pero sólo satisface 1. y 2. (no requiere 3, es decir no requiere inversos)


\end{definition}


\begin{example}

$(\mathbb{Z}, +)$ y $(\mathbb{R} - \{0\}, \cdot)$ son grupos conmutativos.

$(\mathbb{R}^{2 \times 2}, \cdot)$ es un monoide (no conmutativo), donde $ \mathbb{R}^{2 \times 2}$ representa las matrices de $ 2 \times 2$ con coeficientes reales, y $ \cdot$ es la multiplicación de matrices.

\end{example}


\begin{definition}
Un \textbf{anillo} es un conjunto $R$ con dos funciones $ + : R \times R \to R$ y $\cdot : R \times R \to R$, tales que

\begin{enumerate}
\item $(R,+)$ es grupo conmutativo.

\item $(R,\cdot)$ es un monoide.

\item La multiplicación se distribuye sobre la suma, es decir dados $ a,b,c \in R$ se tiene $a \cdot (b + c) = ab + ac$ y $ (b+c) \cdot a = ba + ca$

\end{enumerate}

Si además resulta que la multiplicación es conmutativa, se dice que es un \textbf{anillo conmutativo}.

\end{definition}

\begin{example}

Algunos ejemplos de anillos

\begin{enumerate}

\item $ (\mathbb{R}^{2 \times 2}, +, \cdot)$ es un anillo no conmutativo.

\item $ (\mathbb{Z}, +, \cdot)$ es un anillo conmutativo.

\end{enumerate}
\end{example}

\begin{definition}

Un \textbf{cuerpo} es un conjunto $K$ con dos funciones $+ : K \times K \to K$, y $\cdot : K \times K \to K$ tales que $(K, +, \cdot)$ es un anillo conmutativo y además $(K-\{0\}, \cdot)$ es grupo conmutativo.

\end{definition}

\begin{example}

Algunos ejemplos de cuerpos

\begin{itemize}

\item $ (\mathbb{Q}, +, \cdot)$
\item $ (\mathbb{R}, +, \cdot)$ 
\item $(\mathbb{C}, +, \cdot)$

\end{itemize}

\end{example}

Observar que $ (\mathbb{Z}, +, \cdot)$ no es un cuerpo, en particular $ 2$ no tiene inverso multiplicativo en $ \mathbb{Z}$.

\begin{definition} 

Un \textbf{espacio vectorial} \label{ev} es un conjunto $V$ y un cuerpo $K$ con una función $+ : V \times V \to V$ y una función $ \cdot : K \times V \to V$ tal que

\begin{enumerate}
\item $ (V,+)$ es grupo conmutativo.
\item $ 1 \cdot v = v$ para todo $ v \in V$
\item $ a(bv) = (ab)v$ para todo $ a,b \in K$ y $ v \in V$
\item $ k(v+w) = kv + kw$ para todo $ k \in K$ y $ v,w \in V$
\item $ (k+q)v = kv + qv$ para todo $ k,q \in K$ y $ v \in V$
\end{enumerate}

\end{definition}

\begin{example}
$ V = (\mathbb{R}^n, +, \mathbb{R}, \cdot)$ es un espacio vectorial.  Es el principal espacio vectorial con el que trabajamos en esta materia.
\end{example}

\begin{definition} 

Un \textbf{espacio con producto interno} \label{evpi} es un espacio vectorial $ V$ sobre el cuerpo $ \mathbb{K}$ (que debe ser $ \mathbb{R}$ o $ \mathbb{C}$) y con una función $ \langle - , - \rangle : V \times V \to \mathbb{K}$ tal que para todo $ u, u', v \in V$ y $ \lambda \in \mathbb{K}$ se tiene

\begin{enumerate}
\item $ \langle u + u', v \rangle = \langle u,v \rangle + \langle u', v \rangle $

\item $ \langle \lambda u, v \rangle = \lambda \langle u,v \rangle $

\item $ \langle u,v \rangle = \overline{ \langle v,u \rangle } $

\item $ \langle v,v \rangle > 0$ si $ v \neq 0$
\end{enumerate}
\end{definition}

\begin{example}

En $ \mathbb{R}^n$ podemos definir el producto interno usual como

$$ u \cdot v = \sum_{i=1}^n u_i v_i = u_1 v_1 + u_2 v_2 + \ldots + u_n v_n$$

donde $ u=(u_1, \ldots, u_n)$ y $ v = (v_1, \ldots, v_n) \in \mathbb{R}^n$.

También se lo conoce como producto punto, o producto escalar.

Luego $\mathbb{R}^n$ con el producto $\cdot$ forma un espacio con producto interno.  
\end{example}