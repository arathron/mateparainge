\documentclass[11pt,fleqn]{book} % Default font size and left-justified equations
%\documentclass[a4paper]{article}

\usepackage{tgheros}
\usepackage[T1]{fontenc}
\renewcommand*\familydefault{\sfdefault}

\usepackage{graphicx}
\usepackage{makeidx}

\usepackage[spanish]{babel}
\usepackage[utf8]{inputenc}

\usepackage{fancybox, framed}
\usepackage{enumerate}

\usepackage{amssymb}
\usepackage{amsmath}
% atajos para conjuntos usuales
\newcommand\NN{\mathbb{N}}
\newcommand\ZZ{\mathbb{Z}}
\newcommand\QQ{\mathbb{Q}}
\newcommand\RR{\mathbb{R}}
\newcommand\CC{\mathbb{C}}
\newcommand\FF{\mathbb{F}}


%
% Usando amsthm, es más básico que ntheorem.
%

% Para las secciones Theorem, Proof, Definition 
\usepackage{amsthm}

\theoremstyle{plain}
\newtheorem{theorem}{Teorema}[section]
\newtheorem{lemma}[theorem]{Lema}
\newtheorem{proposition}[theorem]{Proposición}
\newtheorem{property}[theorem]{Propiedad}
\newtheorem{corollary}[theorem]{Corolario}

\theoremstyle{definition}
\newtheorem{definition}[theorem]{Definición}
\newtheorem{observation}[theorem]{Observación}

\theoremstyle{remark}
\newtheorem{example}[theorem]{Ejemplo}
\newtheorem{conjecture}[theorem]{Conjetura}

%\newtheorem*{note}{Nota}



\makeindex

\begin{document}

\title{Análisis Matemático II}
\author{Damián Silvestre}
%\date{}
\maketitle


%%%%%%%%%%%%%%%%%%%%%%%%%%%%%%%%%%%%%%%%%%%%%%%%%%%%%%%%%%%%%%%%%%%%%%%%%%%%%%%
% Análisis en una variable
%
% Copyright (c) 2016 Damián Silvestre. Permission is granted to copy, 
% distribute and/or modify this document under the terms of the 
% GNU Free Documentation License, Version 1.3 or any later version published by
% the Free Software Foundation; with no Invariant Sections, no 
% Front-Cover Texts, and no Back-Cover Texts. 
%
% Details of the GNU FDL can be found here: 
% http://www.gnu.org/licenses/licenses.html
%
%%%%%%%%%%%%%%%%%%%%%%%%%%%%%%%%%%%%%%%%%%%%%%%%%%%%%%%%%%%%%%%%%%%%%%%%%%%%%%%

\part{Análisis en una variable}

\chapter{Primer capítulo}

\section{Primera sección}

\begin{definition}
Primer definición
\end{definition}



\printindex


\end{document}
