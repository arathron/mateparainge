%%%%%%%%%%%%%%%%%%%%%%%%%%%%%%%%%%%%%%%%%%%%%%%%%%%%%%%%%%%%%%%%%%%%%%%%%%%%%%%
% Precálculo
%
% Copyright (c) 2016 Damián Silvestre. Permission is granted to copy, 
% distribute and/or modify this document under the terms of the 
% GNU Free Documentation License, Version 1.3 or any later version published by
% the Free Software Foundation; with no Invariant Sections, no 
% Front-Cover Texts, and no Back-Cover Texts. 
%
% Details of the GNU FDL can be found here: 
% http://www.gnu.org/licenses/licenses.html
%
%%%%%%%%%%%%%%%%%%%%%%%%%%%%%%%%%%%%%%%%%%%%%%%%%%%%%%%%%%%%%%%%%%%%%%%%%%%%%%%

\part{Precálculo}

\chapter{Nociones de lógica}

\section{Lógica proposicional}

\begin{definition}
Una \textbf{proposición} (P) \index{Lógica!Proposición} es un enunciado que tiene un valor de verdad, es decir que puede considerarse como \textbf{verdadero} (\textbf{V}) o \textbf{falso} (\textbf{F}).
\end{definition}

\begin{definition}
Se pueden crear nuevas proposiciones a partir de otras utilizando \textbf{conectivos lógicos}. \index{Lógica!Conectivos lógicos}

Definimos a continuación los principales conectivos lógicos a partir de sus \textbf{tablas de verdad}. \index{Lógica!Tabla de verdad}

\begin{itemize}
\item \textbf{Negación}: \index{Lógica!Conectivos lógicos!Negación}

\begin{tabular}{ |c|c| }
\hline
$P$ & $\neg P$  \\
\hline  
V & F  \\
F & V  \\
\hline
\end{tabular}

Ejemplo: \textbf{No} está lloviendo.

\item \textbf{Conjunción}: \index{Lógica!Conectivos lógicos!Conjunción}

\begin{tabular}{ |c|c|c| }
\hline
$P$ & $Q$ & $P \wedge Q$  \\
\hline  
V & V & V  \\
V & F & F  \\
F & V & F  \\
F & F & F  \\
\hline
\end{tabular}

Está lloviendo \textbf{y} está nublado.

\item \textbf{Disyunción}: \index{Lógica!Conectivos lógicos!Disyunción}

\begin{tabular}{ |c|c|c| }
\hline
$P$ & $Q$ & $P \vee Q$  \\
\hline  
V & V & V  \\
V & F & V  \\
F & V & V  \\
F & F & F  \\
\hline
\end{tabular}

Ejemplo: Está lloviendo \textbf{o} está soleado.

\item \textbf{Disyunción excluyente}: \index{Lógica!Conectivos lógicos!Disyunción excluyente}

\begin{tabular}{ |c|c|c| }
\hline
$P$ & $Q$ & $P \vee Q$  \\
\hline  
V & V & F  \\
V & F & V  \\
F & V & V  \\
F & F & F  \\
\hline
\end{tabular}

Ejemplo: \textbf{O bien} está soleado, \textbf{o bien} está nublado.

\item \textbf{Condicional}: \index{Lógica!Conectivos lógicos!Condicional}

\begin{tabular}{ |c|c|c| }
\hline
$P$ & $Q$ & $P \Rightarrow Q$  \\
\hline  
V & V & V  \\
V & F & F  \\
F & V & V  \\
F & F & V  \\
\hline
\end{tabular}

Ejemplo: Si está soleado, \textbf{entonces} es de día.

\item \textbf{Bicondicional}: \index{Lógica!Conectivos lógicos!Bicondicional}

\begin{tabular}{ |c|c|c| }
\hline
$P$ & $Q$ & $P \Leftrightarrow Q$  \\
\hline  
V & V & V  \\
V & F & F  \\
F & V & F  \\
F & F & V  \\
\hline
\end{tabular}

Ejemplo: Está nublado \textbf{si y sólo si} hay nubes visibles.

\end{itemize}

\end{definition}



\chapter{Números y geometría}

\section{Conjuntos}

\begin{definition}[Conjunto] \index{Conjunto}
Un \textbf{conjunto} es una colección de objetos, tal que si uno tiene un objeto cualquiera, puede determinar si está o no en la colección, o sea si \textbf{pertenece} o no.

Los objetos que pertenecen al conjunto se llaman los \textbf{elementos} del conjunto.

Si $A$ es un conjunto, y $x$ un elemento de $A$, lo denotamos $x \in A$.  Si $y$ no es un elemento de $A$, lo denotamos $y \not\in A$.

Si $A$ y $B$ son conjuntos, para todo $x \in A$ se tiene que $x \in B$, decimos que $A$ es un \textbf{subconjunto} de $B$  y lo denotamos $A \subseteq B$.  Si $A \subseteq B$ y $B \subseteq A$, los conjuntos $A$ y $B$ tienen los mismos elementos, y decimos que los conjuntos son \textbf{iguales}, es decir $A = B$.
\end{definition}

\begin{definition}[Cardinal] \index{Conjunto!Cardinal}
Si un conjunto $A$ tiene una cantidad $n$ de elementos, decimos que su \textbf{cardinal} es $|A| = n$, y que $A$ tiene \textbf{cardinal finito}.

Si no, decimos que $A$ tiene \textbf{cardinal infinito} $|A| = \infty$

\end{definition}


\begin{definition}[Operaciones con conjuntos] \index{Conjunto!Operaciones}
Sean $A$ y $B$ conjuntos.  

La \textbf{unión} \index{Conjunto!Operaciones!Unión} de $A$ con $B$ es

$$A \cup B = \{ x / x \in A \vee x \in B \}$$

La \textbf{intersección} \index{Conjunto!Operaciones!Intersección} de $A$ con $B$ es

$$A \cap B = \{ x / x \in A \wedge x \in B \}$$

La \textbf{diferencia} \index{Conjunto!Operaciones!Diferencia} de $A$ con $B$ es

$$A - B = \{ x \in A / x \not\in B\}$$

Si fijamos un conjunto $U$ como conjunto universal, el \textbf{complemento} \index{Conjunto!Operaciones!Complemento} de cualquier subconjunto $A \subseteq U$ es $U-A$, y lo denotamos por $A^c$

\end{definition}

\begin{definition}[Producto Cartesiano] \index{Conjunto!Producto Cartesiano}
Si $A$ y $B$ son conjuntos, el producto cartesiano de $A$ con $B$ es el conjunto de todos los pares ordenados $(a,b)$ con $a \in A$ y $b \in B$, es decir

$$ A \times B = \{(a,b) : a \in A, b \in B \}$$

Dados $(a,b)$ y $(c,d)$ elementos de $A \times B$, se dice que $(a,b) = (c,d)$ si y sólo si $a = c$ y $b = d$.
\end{definition}


\subsection{Relaciones y Funciones}

\begin{definition}[Relación] \index{Conjunto!Relación}

Dados $A$ y $B$ conjuntos, una \textbf{relación} de $A$ en $B$ es un subconjunto de $A \times B$.

Sea $R$ una relación.  Si $(a,b) \in R$, también lo denotamos $a R b$.

Sea $R$ una relación de $A$ en $A$.  Se dice que es

\begin{itemize}

\item reflexiva: si $xRx$ para todo $x \in A$.

\item simétrica: si $xRy$ implica que $yRx$ para todos $x,y \in A$.

\item antisimétrica: si $xRy$ y $yRx$ implica que $y=x$.

\item transitiva: si $xRy$ y $yRz$ implica que $xRz$.

\end{itemize}

Una relación reflexiva, simétrica y transitiva se dice que es una \textbf{relación de equivalencia}. \index{Conjunto!Relación!Relación de equivalencia}

Una relación reflexiva, antisimétrica y transitiva se dice que es una \textbf{relación de orden} (o de orden parcial). \index{Conjunto!Relación!Relación de orden}

Un conjunto en el cual se definió un orden se dice que es un \textbf{conjunto ordenado}, y si el orden es total se dice que es un \textbf{conjunto totalmente ordenado}.

En un conjunto ordenado, puede haber elementos no comparables, es decir elementos donde ni $xRy$ ni $yRx$.  Si en un conjunto ordenado todos los elementos son comparables decimos que se trata de un \textbf{orden total} (o lineal).

\end{definition}

\begin{definition}[Máximo, mínimo, cotas, supremo, etc]

Sea $A$ un conjunto ordenado.  

\begin{itemize}

\item Si existe $z \in A$ tal que para todo $x \in A$ se tiene que $x \leq z$ se dice que $z$ es el \textbf{máximo} \index{Orden!máximo} de $A$.

\item Si existe $z \in A$ tal que para todo $x \in A$ se tiene que $x \geq z$ se dice que $z$ es el \textbf{mínimo} \index{Orden!mínimo} de $A$.

\item Si para $z \in A$ se cumple que para todo $x \geq z$ se tiene que $x = z$ se dice que $z$ es un \textbf{elemento maximal} \index{Orden!maximal} de $A$.

\item Si para $z \in A$ se cumple que para todo $x \leq z$ se tiene que $x = z$ se dice que $z$ es un \textbf{elemento minimal} \index{Orden!minimal} de $A$.

\end{itemize}

Sea $B \subseteq A$, entonces

\begin{itemize}

\item Si existe $z \in A$ tal que para todo $b \in B$ se tiene que $b \leq z$, decimos que $z$ es una \textbf{cota superior} \index{Orden!cota superior} de $B$.

\item Si existe $z \in A$ tal que para todo $b \in B$ se tiene que $b \geq z$, decimos que $z$ es una \textbf{cota inferior} \index{Orden!cota inferior} de $B$.

\item Si el conjunto de cotas superiores de $A$ tiene un mínimo $z$, decimos que $z$ es el \textbf{supremo} \index{Orden!supremo} de $A$.

\item Si el conjunto de cotas inferiores de $A$ tiene un máximo $z$, decimos que $z$ es el \textbf{ínfimo} \index{Orden!ínfimo} de $A$.

\end{itemize}

\end{definition}

\begin{definition}[Función] \index{Función} \label{funcion}

Una \textbf{función} de $A$ en $B$ es una relación tal que para todo $a \in A$ existe $b \in B$ de forma tal que $(a,b) \in R$, y además dicho $b$ es único, es decir que si se tiene $(a,b) \in R$ y $(a,c) \in R$, entonces $b=c$.

Se la denota de la forma $f : A \to B$, y si se tiene $(a,b) \in f$, también se escribe $f(a) = b$, y se dice que la imagen de $a$ por $f$ es $b$.

El conjunto $A$ se llama \textbf{dominio} \index{Función!Dominio} y el conjunto $B$ se llama \textbf{codominio} \index{Función!Codominio} de la función.

Dado $C \subseteq A$, el conjunto $f(C) = \{b \in B : \exists c \in C, f(c) = b\}$ se llama imagen de $C$ por $f$.  El conjunto $f(A)$ se llama la \textbf{imagen} \index{Función!Imagen} de $f$, y se denota $Im(f)$.  Es decir

$$ Im(f) = \{ b \in B : \exists a \in A, f(a) = b \} $$

La \textbf{gráfica} de la función es la relación funcional que las define, es decir

$$ Graf(f) = \{ (x,y) \in A \times B : y = f(x) \} $$

\end{definition}



\section{Conjuntos numéricos}

\begin{definition}[Números reales] \index{Conjunto!Números reales}
El principal conjunto con el que vamos a trabajar es el conjunto de los \textbf{números reales}, que denotaremos $\mathbb{R}$.  En este conjunto están definidas las operaciones suma $+$ y producto $\cdot$ con las siguientes propiedades:

Para todo $x,y,z \in \mathbb{R}$

\begin{itemize}

\item Asociativa para la suma

$x+(y+z) = (x+y)+z$

\item Neutro para la suma: existe $0 \in \mathbb{R}$

$x+0 = 0+x = x$

\item Inverso para la suma: para todo $x$ existe $-x \in \mathbb{R}$ tal que

$x + (-x) = (-x) + x = 0$

\item Conmutatividad de la suma

$x+y = y+x$

\item Asociatividad del producto

$x(yz) = (xy)z$

\item Neutro del producto: existe $1 \in \mathbb{R}$ tal que

$1 \cdot x = x \cdot 1 = x$

\item Inverso del producto: para todo $x \neq 0$ existe $x^{-1}$ tal que

$x \cdot x^{-1} = x^{-1} \cdot x = 1$

\item Conmutatividad del producto

$xy = yx$

\item Distributiva

$x(y+z) = xy + xz$

$(x+y)z = xz + yz$

\item Se tiene un orden total $\leq$ tal que si $x \leq y$ y $z \in \mathbb{R}$ entonces

$x+z \leq y+z$

y además si $z \geq 0$ entonces

$xz \leq yz$ 

\item Todo subconjunto $A \subseteq \mathbb{R}$ acotado superiormente tiene un supremo.

\end{itemize}
\end{definition}

\begin{definition}[Números naturales] \index{Conjunto!Números naturales}
Sea $A \subseteq \mathbb{N}$.  Decimos que $A$ es un \textbf{conjunto inductivo} si cumple las siguientes

\begin{itemize}

\item $1 \in \mathbb{N}$

\item Si $n \in \mathbb{N}$ entonces $n+1 \in \mathbb{N}$

\end{itemize}

Definimos el conjunto de los \textbf{números naturales} $\mathbb{N}$ como la intersección de todos los conjuntos inductivos.  Es decir sus elementos son 

$$ \mathbb{N} = \{1,2,3,4,5, \ldots \} $$
\end{definition}

\begin{definition}[Números enteros] \index{Conjunto!Números enteros}
El conjunto de los \textbf{números enteros} contiene a los naturales, y le agrega el $0$ y los inversos aditivos, es decir

$$ \mathbb{Z} = \{\ldots, -4,-3,-2,-1,0,1,2,3, \ldots \}$$
\end{definition}

\begin{definition}[Números racionales] \index{Conjunto!Números racionales}
El conjunto de los \textbf{números racionales} satisface todas las propiedades de los reales salvo la del axioma del supremo.  Son los que podemos escribir como fracciones

$$ \mathbb{Q} = \{ \frac{p}{q} : p,q \in \mathbb{Z}, q \neq 0 \} $$

y se tiene que $\frac{a}{b} = \frac{c}{d}$ si y sólo si $ad = cb$.
	
Entre dos racionales siempre existe otro, por ejemplo sean $x,y \in \mathbb{Q}$ con $x < y$.  Entonces $z = \frac{x+y}{2}$ es tal que $x < z < y$.  Lo mismo puede decirse entre dos reales $x<y$ en general.
\end{definition}

\begin{definition}[Números complejos] \index{Conjunto!Números complejos}
El conjunto de los \textbf{números complejos} consisten en agregarle a los reales un nuevo número $i \in \mathbb{C}$ tal que $i^2 = -1$

$$ \mathbb{C} = \{ a + b i : a,b \in \mathbb{R}, i^2 = -1 \}$$

Si identificamos $a+bi$ con el par ordenado $(a,b)$ podemos identificar $\mathbb{C}$ con $\mathbb{R}^2$ con la siguiente regla de multiplicación

$$ (a+bi)(c+di) = (ac - bd) + (ad + bc) i$$

o sea

$$ (a,b)(c,d) = (ac - bd, ad + bc)$$
\end{definition}



\section{Divisibilidad en $\mathbb{Z}$}

\begin{definition}[Divisibilidad] \index{Conjunto!Números enteros!Divisibilidad}
Sean $a,b,c \in \mathbb{Z}$ tales que $a = bc$.  Decimos entonces que $b$ \textbf{divide} a $a$ y lo denotamos $b | a$.  También decimos que $b$ es un \textbf{divisor} de $a$, o que $a$ es divisible por $b$.  El conjunto de divisores de $a$ lo denotamos $Div(a)$

Si $b$ no divide a $a$, es decir no existe $c$ tal que $bc = a$, lo denotamos $b \not| a$.

Si $2|x$ decimos que $x$ es \textbf{par}, de lo contrario $2 \not| x$ y decimos que $x$ es \textbf{impar}.
\end{definition}

\begin{definition}[Número primo] \index{Conjunto!Números enteros!Número primo}
Todo $x \in \mathbb{Z}$ es divisible por $1,-1,x,-x$.  Si $x \neq 0,1,-1$ y sus únicos divisores son $1,-1,x,-x$, decimos que $x$ es un \textbf{número primo}.
\end{definition}

\begin{definition}[MCD, MCM, coprimos] \index{Conjunto!Números enteros!MCD,MCM,coprimos}
Sean $x,y \in \mathbb{Z}$.  El \textbf{máximo común divisor} (MCD) entre $x$ e $y$ es el máximo de los divisores comunes, y lo denotamos $(x:y)$

El \textbf{mínimo común múltiplo} (MCM) entre $x$ e $y$ es el mínimo de los múltiplos comunes, y lo denotamos $[x:y]$

Se verifica la siguiente propiedad: $[x:y] (x:y) = x \cdot y$

Decimos que $x$ e $y$ son \textbf{coprimos} si y sólo si $(a:b) = 1$
\end{definition}

\begin{theorem}[Teorema fundamental de la aritmética (TFA)] \index{Conjunto!Números enteros!TFA}
Para todo $x \in \mathbb{Z}$, tal que $x \neq 0, 1, -1$, existen números primos distintos $p_1, p_2, \ldots, p_n \in \mathbb{Z}$ positivos, y números naturales $a_1, a_2, \ldots, a_n \in \mathbb{N}$ de forma tal que

$$ x = \pm p_1^{a_1} p_2^{a_2} \ldots p_n^{a_n} $$

donde $\pm$ es el signo de $x$, y además dicha escritura es única salvo el orden de los factores.
\end{theorem}

\begin{observation}[MCD y MCM usando TFA]
Una regla para determinar $(x:y)$ consiste en descomponer $x$ e $y$ en sus factores primos, y luego multiplicar los factores comunes con su menor exponente.  Por ejemplo para calcular $(12,28)$ primero descompongo $12 = 2^3 \cdot 3$ y $28 = 2^2 \cdot 7$, entonces $(12:28) = 2^2 = 4$.
	
Y para determinar $[x:y]$ multiplicamos los factores comunes con su mayor exponente.  Por ejemplo $[12:28] = 2^3 \cdot 3 \cdot 7 = 168$.	
\end{observation}

\section{Expresión decimal de un racional} \index{Conjunto!Números racionales!Expresión decimal}

Sea $\frac{a}{b} \in \mathbb{Q}$, entonces al efectuar la división podemos expresarlo en forma decimal, con una cantidad finita de decimales, o infinita periódica.

Ejemplos:

$\frac{1}{2} = 0.5$

$\frac{3}{2} = 1.5$

$\frac{1}{3} = 0.333333\ldots = 0.\overline{3}$

$\frac{3549}{990} = 3.5\overline{84}$

En este último ejemplo, para pasar de la notación decimal a la fraccionaria, realizamos la siguiente operación

$ \frac{3584 - 35}{990} = \frac{3549}{990} = \frac{1183}{330}$

\section{Intervalos}

\begin{definition}[Intervalo] \index{Conjunto!Números reales!Intervalo}

Un intervalo es un subconjunto $I \subseteq \RR$ tal que para todos $x,y \in I$ y todo $x < c < y$ se tiene que $c \in I$.

Los intervalos los podemos clasificar de la siguiente manera.  Sean $a,b \in \RR$ con $a<b$ entonces tenemos

\begin{itemize}

\item El intervalo finito abierto 

$(a,b) = \{ x \in \mathbb{R} : a < x < b \}$

\item El intervalo finito cerrado

$[a,b] = \{ x \in \mathbb{R} : a \leq x \leq b \}$

\item Los intervalos finitos semiabiertos

$(a,b] = \{ x \in \mathbb{R} : a < x \leq b \}$

$[a,b) = \{ x \in \mathbb{R} : a \leq x < b \}$

\item Los intervalos infinitos abiertos

$(a, +\infty) = \{ x \in \mathbb{R} : x > a \}$

$(+\infty, a) = \{ x \in \mathbb{R} : x < a \}$

\item Los intervalos infinitos cerrados

$[a, +\infty) = \{ x \in \mathbb{R} : x \geq a \}$

$(+\infty, a] = \{ x \in \mathbb{R} : x \leq a \}$

\end{itemize}
\end{definition}



\section{Valor absoluto}

\begin{definition}[Módulo o valor absoluto] \label{funcion_modulo} \index{Función!Ejemplos!Función módulo}
El \textbf{valor absoluto} (o módulo) es la siguiente función $| \cdot | : \mathbb{R} \to \mathbb{R}$

$$ |x| = \begin{cases} x & \textrm{ si } x \geq 0 \\ -x & \textrm{ si } x < 0 \end{cases} $$

\end{definition}

\begin{property}[del valor absoluto]
Se verifican: 

\begin{itemize}
\item $ |x| \geq 0 $ para todo $x \in \mathbb{R}$
\item Si $ |x| = 0$ entonces $x = 0$
\item $ |ab| = |a| |b|$
\item $|a+b| \leq |a| + |b|$

\item $|a| < b$ si y sólo si $-b < a < b$

\item $|a| > b$ si y sólo si $a > b$ o bien $a < -b$

\item $|a| = b$ si y sólo si $a=b$ o bien $a = -b$

\end{itemize}
\end{property}

\begin{definition}[Distancia] \index{Conjunto!Números reales!distancia}
Sean $x, y \in \mathbb{R}$, definimos la \textbf{distancia} entre $x$ e $y$ como 

$$d(x,y) = |y - x|$$

\end{definition}

\begin{property}[de la distancia] Se verifican para todo $x,y,z \in \mathbb{R}$

\begin{itemize}
\item $d(x,y) \geq 0$
\item $d(x,y) = 0$ si y sólo si $x=y$
\item $d(x,y) = d(y,x)$
\item $d(x,z) \leq d(x,y) + d(y,z)$
\end{itemize}

\end{property}


\section{Sobre exponentes y raíces}

\begin{definition}[Potencia $n$-ésima] \index{Conjunto!Números reales!Potencia y raíz}
Si $x \in \mathbb{R}$ y $n \in \mathbb{N}$ definimos $x$ a la $n$-ésima \textbf{potencia} como

$$x^n = \underbrace{x \cdot x \cdot \ldots \cdot x}_{n \ veces} = y$$

A $x$ lo llamamos \textbf{base}, a $n$ lo llamamos el \textbf{exponente}, y a $y$ potencia.
\end{definition}

\begin{observation}
Si $n$ es par, $y \geq 0$.  Si $n$ es impar, el signo de $x$ y de $y$ coinciden.
\end{observation}	

\begin{definition}[Raíz $n$-ésima]
Ahora al revés, supongamos que conocemos $y \in \mathbb{R}$, y $n \in \mathbb{N}$, llamamos \textbf{raíz $n$-ésima} de $x$, y lo denotamos $\sqrt[n]{x} = y$, al número $y$ tal que $y^n = x$.
\end{definition}	

\begin{observation}
Por la observación anterior, si $x < 0$ y \textbf{$n$ es par}, no queda definida en $\mathbb{R}$ dicha raíz.  Por ejemplo $\sqrt{-1} \not\in \mathbb{R}$.  Resulta que en los otros casos la raíz existe y es única (en $\mathbb{R}$).
\end{observation}

\begin{property}
Si $x,y \in \mathbb{R}$ y $n$ es impar

$$ (xy)^n = x^n y^n $$

Si $x,y \geq 0$ y $n$ es par también se cumple dicha igualdad.
\end{property}

\begin{definition}[Potenciación con números racionales y reales]
Extendemos la definición de potencia primero a los racionales mediante

\begin{itemize}
\item $x^{1/n} = \sqrt[n]{x}$ 
\item $x^{-n} = \frac{1}{x^n}$
\item $x^{p/q} = \sqrt[q]{x^p}$
\end{itemize}
	
La última definición nos permite potenciar con exponentes racionales.  

Si $x,n \in \mathbb{R}$, $x,n>0$, es posible definir $x^n$.  Por ahora este caso lo trabajamos sólamente con la calculadora.
	
\end{definition}


\section{Teorema de Pitágoras}

\begin{theorem}[Pitágoras] \index{Geometría!Teorema de Pitágoras}
Sea un triángulo rectángulo como el de la figura en color amarillo.

Llamamos $a$ y $b$ a sus catetos, y $c$ a su hipotenusa.  Entonces 

$$ c^2 = a^2 + b^2 $$
\end{theorem}

\begin{figure}[h]
\centering\includegraphics[scale=1]{images/01_precalculo/Pythagore.jpg}
\caption{Pitagoras}
\end{figure}

\begin{proof}
El área del cuadrado grande es 

$A = (a+b)(a+b) = a^2 + 2ab + b^2$

El área del cuadrado chico es $c^2$.  Pero además el área del cuadrado grande es igual a la del cuadrado chico mas cuatro veces el área del triángulo amarillo.  Es decir

$ a^2 + 2ab + b^2 = c^2 + 4 \frac{ab}{2} $

$ a^2 + 2ab + b^2 = c^2 + 2 ab $

$ a^2 + b^2 = c^2 $

como queríamos demostrar.
\end{proof}



\section{Números irracionales} \index{Conjunto!Números reales!Irracionales}

Veamos que algunos números reales no son racionales.  A los números reales que no son racionales los llamamos \textbf{irracionales}.

Supongamos que tenemos un cuadrado de lado 1 como en la figura.

\begin{figure}[h]
\centering\includegraphics[scale=0.5]{images/01_precalculo/unit-square-with-diagonal.jpg}
\caption{Diagonal de un cuadrado}
\end{figure}

Queremos determinar el largo $c$ de su diagonal.  Por el teorema de pitágoras sabemos que $c^2 = 1^2 + 1^2 = 2$, es decir que $c = \sqrt{2}$.  

\begin{theorem}
$\sqrt{2} \not\in \QQ$
\end{theorem}

\begin{proof}
Supongamos que $\sqrt{2}$ es un número racional, es decir $\sqrt{2} = p/q$, con $p,q \in \mathbb{N}$ y coprimos, es decir $(p:q) = 1$, de forma que la fracción esté expresada en forma irreducible.

Elevando al cuadrado se tiene

$ \frac{p^2}{q^2} = 2$

$ p^2 = 2 q^2$

O sea que $p^2$ es par.  Por el teorema fundamental de la aritmética esto implica que $p$ es par.  O sea que $p = 2m$.  Luego

$ 4m^2 = 2q^2$

$ 2m^2 = q^2$

Con un razonamiento análogo al anterior vemos que $q$ es par.  Esto es absurdo pues $p$ y $q$ eran coprimos.  El absurdo provino de suponer que $\sqrt{2}$ es racional.  Por lo tanto $\sqrt{2}$ no es racional.
\end{proof}

\begin{problem}
Prueba gráfica de que $32.5 = 31.5$.  ¿Cómo es posible?
\end{problem}

\begin{figure}[h]
\centering\includegraphics[scale=0.6]{images/01_precalculo/ilusion_triangulo.jpg}
\caption{Ilusión triángulo}
\end{figure}

Rta: Es una ilusión óptica.  El de arriba pareciera ser un triángulo de area $ \frac{13 \cdot 5}{2} = 32.5$.  Y el de abajo es reacomodar las fichas pero parece un triángulo con un cuadrado menos, es decir una region con área $31.5$.  

Pero en realidad ninguna de las dos figuras es un triángulo.  Basta ver que los ángulos de los triángulos rojo y azul no son iguales, sus tangentes (opuestos/adyancete) son $3/8 = 0.375 \neq 0.4 = 2/5 $.



\chapter{Polinomios}

\begin{definition}[Monomio] \index{Polinomio!Monomio}
Un \textbf{monomio} es una expresión en la que intervienen números reales, y variables, sólamente relacionadas por multiplicación (y potenciación con exponente natural).  

El \textbf{grado} del monomio es la suma de los exponentes de las variables.
\end{definition}

\begin{example}
Algunos monomios y sus grados:

\begin{itemize}
\item $12x^2y$ es de grado 3
\item $\pi z x^3$ es de grado 4
\item $-25 u^2$ es de grado 2
\end{itemize}
\end{example}

\begin{definition}[Polinomio] \index{Polinomio}
Un \textbf{polinomio} es una suma de monomios, y su \textbf{grado} es el grado del monomio de mayor \textbf{grado} que interviene en la suma.
\end{definition}

\begin{example}

Algunos polinomios y sus grados

\begin{itemize}
\item $12x^2y + \pi z x^3 $ es de grado 4
\item $3xy + z^2 - 3uv + x^3$ es de grado 3
\end{itemize}
	
\end{example}



\section{Polinomios en una indeterminada}

\begin{definition}[Polinomio en una variable] \index{Polinomio!Real}
En esta parte, sólamente vamos a trabajar con polinomios en una sóla variable ó indeterminada.  Al conjunto de todos los polinomios (con coeficientes en $\RR$) \textbf{en una sola variable} $x$ lo denotamos $\RR[x]$

Siempre podemos expresar un polinomio $p \in \RR[x]$ en la forma $p = \sum_{i=0}^n a_i x_i$, es decir

$$ p = a_0 + a_1 x + a_2 x^2 + \ldots + a_n x^n $$

con $a_n \neq 0$, y el grado del polinomio lo denotamos $gr(p) = n$

El polinomio nulo $p=0$ es el único polinomio que no tiene grado.

Decimos que el polinomio $p$ es \textbf{mónico} si $a_n = 1$.

Dos polinomios $p$  y $q$ son \textbf{iguales} si y sólo si sus coeficientes son iguales.  Es decir si $p = \sum_{i=0}^n a_i x^i$ y $q = \sum_{j=0}^m b_j x^j$ entonces $p=q$ si y sólo si $n=m$ y $a_i = b_i$ para todo $0 \leq i \leq n$
\end{definition}

\begin{example}
Algunos polinomios de $\RR[x]$
\begin{itemize}

\item $3x^2 + 2x -3$ es de grado 3
\item $\pi x + 8x^5 $ es de grado 5
\end{itemize}

\end{example}

\subsection{Operaciones con polinomios}

\begin{definition}[Suma y producto de polinomios] \index{Polinomio!Real!Suma y producto}
	
Sean $p = \sum_{i=0}^n a_i x^i$ y $q = \sum_{j=0}^m b_j x^j$.

Sea $r = max\{ n,m \}$

Entonces la \textbf{suma} la definimos como

$$ p+q = \sum_{k=0}^r (a_k+b_k) x^k $$

es decir se suma coeficiente a coeficiente.

El \textbf{producto} lo definimos como

$$ p \cdot q = \sum_{k=0}^{n+m} c_k x^k $$

donde 

$$ c_k = \sum_{i+j=k} a_i b_j $$

\end{definition}

\begin{observation}
Si $gr(p) = n$ y $gr(q) = m$ entonces $gr(pq) = n+m$
\end{observation}


\begin{definition}[Divisibilidad] \index{Polinomio!Real!Divisibilidad}
Si $p,q,r \in \mathbb{R}[x]$ y $p = qr$, decimos que $q$ \textbf{divide} a $p$, y lo denotamos $q|p$.
\end{definition}

\subsection{Algoritmo de división}

\begin{theorem}[Algoritmo de la división] \index{Polinomio!Real!Algoritmo de la división}
Si $p,q \in \mathbb{R}[x]$ entonces existen únicos polinomios $c,r$ tales que

$$ p = q \cdot c + r$$

con $r=0$ o $gr(r) < gr(q)$

Decimos que $p$ es el \textbf{dividendo}, $q$ el \textbf{divisor}, $c$ el \textbf{cociente}, y $r$ el \textbf{resto}.
\end{theorem}

\begin{definition}[Especialización] \index{Polinomio!Real!Especialización}
Sea $p \in \mathbb{R}[x]$, de forma $p = \sum_{i=0}^n$, y sea $\alpha \in \mathbb{R}$.  La \textbf{especialización} de $p$ en $\alpha$ es el número real

$$ p(\alpha) = \sum_{i=0}^n a_i \alpha^i $$
\end{definition}

\begin{definition}[Raíces y multiplicidad] \index{Polinomio!Real!Raíz y multiplicidad}

Si $p(\alpha) = 0$ se dice que $\alpha$ es una \textbf{raíz} (o un cero) de $p$.  

Decir que $\alpha \in \mathbb{R}$ es raíz de $p$ equivale a decir que $(x-\alpha)$ divide a $p$

Sea $\alpha \in \mathbb{R}$ raíz de $p \in \mathbb{R}[x]$, y sea $a \in \mathbb{N}$ tal que $(x-\alpha)^a | p$ pero $(x-\alpha)^{a+1} \not| p$.  Decimos que $a$ es la \textbf{multiplicidad} de $\alpha$ como raíz de $p$.

Si $a = 1$ se dice que $\alpha$ es \textbf{raíz simple} de $p$, sinó, $a>1$ y se dice que $\alpha$ es \textbf{raíz múltiple} de $p$.
\end{definition}


\begin{theorem}[del resto] \index{Polinomio!Real!Teorema del resto}
Sean $p, q \in \mathbb{R}[x]$, con $q = x - \alpha$.  El resto de dividir $p$ por $q$ es igual a $p(\alpha)$
\end{theorem}

\begin{proof}
Por el algoritmo de la división existen $c$ y $r$ tales que $p(x) = (x-\alpha) c(x) + r(x)$ con $gr(r) < gr(q) = 1$ o $r=0$.  Es decir $r$ constante.

Especializando en $\alpha$ obtenemos $p(\alpha) = r$.
\end{proof}

\begin{definition}[Polinomio irreducible] \index{Polinomio!Real!Irreducible}
Sea $p \in \mathbb{R}[x]$.  Decimos que $p$ es un polinomio \textbf{irreducible} si sus únicos divisores son los polinomios constantes y los polinomios de la forma $c \cdot p$ con $c$ constante.
\end{definition}

\begin{theorem}[Teorema Fundamental de la Aritmética para polinomios] \index{Polinomio!Real!TFA para polinomios}

Sea $p \in \mathbb{R}[x]$.  Entonces existen únicos $c \in \mathbb{R}$, polinomios irreducibles distintos $p_1, p_2, \ldots, p_n \in \mathbb{R}[x]$, y $a_1, a_2, \ldots, a_n \in \mathbb{N}$ de forma que

$$ p = c p_1^{a_1} p_2^{a_2} \ldots p_n^{a_n} $$

\end{theorem}

\begin{theorem}[Gauss] \index{Polinomio!Real!Teorema de Gauss}
Sea $P \in \mathbb{Z}[x]$, $p = a_0 + a_1x + a_2x^2 + \ldots + a_nx^n$, con $a_i \in \mathbb{Z}$ para $0 \leq i \leq n$, y $a_n \neq 0$.  Si $\alpha = \frac{p}{q} \in \mathbb{Q}$ es raíz de $P$, con $(p:q) = 1$, entonces $q | a_n$ y $p | a_0$.
\end{theorem}

\begin{proof}
Como $\alpha = p/q$ es raíz, se tiene que $P(p/q) = 0$, o sea

$a_0 + a_1 \frac{p}{q} + a_2 \frac{p^2}{q^2} + \ldots + a_{n-1} \frac{p^{n-1}}{q^{n-1}} + a_n \frac{p^n}{q^n} = 0$

multiplico por $q^n$

$a_0 q^n + a_1 p q^{n-1} + a_2 p^2 q^{n-2} + \ldots + a_{n-1} p^{n-1} q +  a_n p^n = 0$

saco factor común $p$

$a_0 q^n + p [a_1 q^{n-1} + a_2 p^1 q^{n-2} + \ldots + a_n p^{n-1}] = 0$

Como $p$ divide a $0$ y al segundo sumando, también divide a $a_0 q^n$, pero como $(p:q) = 1$ se tiene que $p | a_0$

Similarmente, sacando factor $q$

$q[a_0 q^{n-1} + a_1 p q^{n-2} + a_2 p^2 q^{n-3} + \ldots a_{n-1} p^{n-1}] + a_n p^n = 0$

Como $q$ divide a $0$ y al primer sumando, se tiene que $q$ divide a $a_n p^n$.  Pero como $(p:q) = 1$, se tiene que $q | a_n$.
\end{proof}

\begin{proposition}[Ecuación de primer grado] \index{Polinomio!Real!Ecuación lineal}
Una ecuación de \textbf{primer grado} (o \textbf{ecuación lineal}) es de la forma

$$ ax + b = 0 $$

con $a \neq 0$, y su única solución es $x = -b/a$.
\end{proposition}

\begin{proposition}[Ecuación de segundo grado] \index{Polinomio!Real!Ecuación cuadrática}
Una ecuación de \textbf{segundo grado} (o \textbf{ecuación cuadrática}) es de la forma

$$ ax^2 + bx + c = 0 $$

con $a \neq 0$.  Sus soluciones (en $\CC$) son dadas por la \textbf{fórmula resolvente}: \label{formula_resolvente}

$$ x_{1,2} = \frac{-b \pm \sqrt{b^2 - 4ac}}{2a}  $$

\end{proposition}

\begin{proof}
	
Divido por $a$

$ x^2 + \frac{b}{a}x + \frac{c}{a} = 0 $

completo cuadrado, es decir busco $\alpha, \beta \in \mathbb{R}$ tal que $(x-\alpha)^2 + \beta = x^2 + \frac{b}{a}x$, es decir

$ x^2 - 2\alpha x + \alpha^2 + \beta = x^2 + \frac{b}{a}x $

entonces resulta que $\alpha = \frac{-b}{2a}$ y $\beta = - \frac{b^2}{4a^2}$.  Luego

$ x^2 + \frac{b}{a}x + \frac{c}{a} = (x + \frac{b}{2a} )^2 - \frac{b^2}{4a^2} + \frac{c}{a}= 0 $

$(x + \frac{b}{2a} )^2 =  \frac{b^2}{4a^2} - \frac{c}{a} $

$(x + \frac{b}{2a} )^2 =  \frac{b^2}{4a^2} - \frac{4ac}{4a^2} $

$(x + \frac{b}{2a} )^2 =  \frac{b^2 - 4ac}{4a^2} $

Llamamos \textbf{discriminante} a $\triangle = b^2 - 4ac$

Debe darse uno de tres casos

\begin{itemize}

\item Si $\triangle > 0$, tenemos dos raíces reales y distintas

$ |x + \frac{b}{2a}| = \frac{\sqrt{ \triangle }}{2a}$

$$ x_{1,2} = \frac{-b}{2a} \pm \frac{\sqrt{ \triangle }}{2a} = \frac{-b \pm \sqrt{b^2 - 4ac}}{2a}$$

\item Si $\triangle = 0$, se tiene una sola raíz real doble.

$$ x = \frac{-b}{2a}$$

\item Si $\triangle < 0$, la ecuación no tiene raíces reales.  Se tienen dos raíces complejas conjugadas

$$x_{1,2} = \frac{-b \pm i \sqrt{4ac - b^2}}{2a}$$

\end{itemize}

En los tres casos se verifica

$$ x_{1,2} = \frac{-b \pm \sqrt{b^2 - 4ac}}{2a}  $$

\end{proof}




\chapter{Sistemas de ecuaciones lineales}

\begin{definition}[Sistema de ecuaciones] \index{Sistema de ecuaciones}
Un \textbf{sistema de ecuaciones} $Ax=k$ es una expresión de la forma

$$ S = \begin{cases}\begin{array}{lcl} 
a_{11} x_1 + a_{12} x_2 + \ldots + a_{1n} x_n & = & k_1 \\
a_{21} x_1 + a_{22} x_2 + \ldots + a_{2n} x_n & = & k_2 \\
& \ldots \\
a_{m1} x_1 + a_{m2} x_2 + \ldots + a_{mn} x_n & = & k_m
\end{array} \end{cases} $$

con $a_{ij} \in \mathbb{R}$ para $1 \leq i \leq m$, $1 \leq j \leq n$, y con $k_i \in \mathbb{R}$ para $1 \leq i \leq m$.

O dicho matricialmente $Ax = k$ con

$$ \underbrace{ \begin{pmatrix} 
a_{11} & a_{12} & \ldots & a_{1n} \\
a_{21} & a_{22} & \ldots & a_{2n} \\
\vdots & & & \\
a_{m1} & a_{m2} & \ldots & a_{mn} 
\end{pmatrix}}_A \underbrace{\begin{pmatrix} x_1 \\ x_2 \\ \vdots \\ x_n \end{pmatrix}}_x = \underbrace{ \begin{pmatrix} k_1 \\ k_2 \\ \vdots \\ k_m \end{pmatrix} }_k $$

La \textbf{matriz asociada} a dicho sistema es

$$ \begin{pmatrix} 
a_{11} & a_{12} & \ldots & a_{1n} & | & k_1 \\
a_{21} & a_{22} & \ldots & a_{2n} & | & k_2 \\
\vdots & & & & | & \\
a_{m1} & a_{m2} & \ldots & a_{mn} & | & k_m 
\end{pmatrix}$$

El \textbf{conjunto de soluciones} del sistema es $S = \{ x \in \RR^n : Ax = k \}$

Dos sistemas se dicen \textbf{equivalentes} si tienen el mismo conjunto de soluciones.  

\end{definition}

\begin{observation}[Operaciones elementales] \index{Sistema de ecuaciones!Operaciones elementales}
	
Siempre se puede llevar un sistema a otro equivalente realizando las siguientes \textbf{operaciones elementales}

\begin{itemize}
\item Permutar ecuaciones / permutar filas.
\item Multiplicar una ecuación/una fila por un escalar $\lambda \neq 0$.
\item A una fila sumarle un múltiplo de otra.
\end{itemize}

\end{observation}

\begin{definition}[Clasificación según soluciones] 
Dado un sistema $S$, entonces se da uno (y solo uno) de los siguientes casos:

\begin{itemize}
\item $\# S = 1$, se lo llama \textbf{Sistema Compatible Determinado} (SCD), y existe una única solución. \index{Sistema de ecuaciones!Clasificación!SCD}
\item $\# S > 1$, se lo llama \textbf{Sistema Compatible Indeterminado} (SCI), y existe más de una solución (y por lo tanto infinitas). \index{Sistema de ecuaciones!Clasificación!SCI}
\item $\# S = 0$, se lo llama \textbf{Sistema Incompatible} (SI), y no tiene ninguna solución. \index{Sistema de ecuaciones!Clasificación!SI}
\end{itemize}
\end{definition}

\subsection{Método de eliminación de Gauss} \index{Sistema de ecuaciones!Método de eliminación de Gauss}

Consiste en realizar las operaciones elementales hasta llevar el sistema a uno equivalente pero que sea triangular inferior, es decir de la forma

$$ \begin{pmatrix} 
a_{11} & a_{12} & \ldots & a_{1n} & | & k_1 \\
0 & a_{22} & \ldots & a_{2n} & | & k_2 \\
0 & 0 & \vdots & \vdots & | & \vdots \\
0 & 0 & 0 & a_{mn} & | & k_m 
\end{pmatrix}$$

Luego reemplazando de las últimas ecuaciones hacia las primeras encontramos todas las soluciones posibles, si es que existen.



\chapter{Funciones}

Recordar la definición de función (Ver \ref{funcion}).

\begin{definition}[Función real] \index{Función!Real}
Una función $f$ se dice que es una \textbf{función real} si es de la forma $f : A \to \mathbb{R}$.  

Un \textbf{cero} \index{Función!Real!Cero o raíz} (o raíz) de $f$ es un $x \in A$ tal que $f(x) = 0$.

Denotamos al conjunto de ceros (o \textbf{conjunto de nivel 0}) de $f$ como

$$ C_0(f) = \{x \in A / f(x) = 0 \} $$

\end{definition}

\begin{definition}[Operaciones con funciones reales] \index{Función!Real!Operaciones}

Sean $f, g : A \to \mathbb{R}$.  Entonces

\begin{itemize}
\item $f+g : A \to \mathbb{R}$ se define de forma tal que $(f+g)(x) = f(x) + g(x)$
\item $f-g : A \to \mathbb{R}$ se define de forma tal que $(f-g)(x) = f(x) - g(x)$
\item $f \cdot g : A \to \mathbb{R}$ se define de forma tal que $(f \cdot g)(x) = f(x) \cdot g(x)$
\item $f/g : A - C_0(g) \to \mathbb{R}$ se define de forma tal que $(f / g)(x) = f(x) / g(x)$
\end{itemize}
\end{definition}

\begin{definition}[Función par, impar, monótona] \index{Función!Real!Par, impar, monótona}

Sea $f : A \subseteq \mathbb{R} \to \mathbb{R}$, se dice que

\begin{itemize}
\item $f$ es \textbf{par} si $f(x) = f(-x)$ para todo $x \in A$.
\item $f$ es \textbf{impar} si $f(x) = -f(-x)$ para todo $x \in A$.
\item $f$ es \textbf{monótona estríctamente creciente} (o estríctamente creciente), si $x < y$ implica que $f(x) < f(y)$ para todo $x,y \in A$.
\item $f$ es \textbf{monótona estríctamente decreciente} (o estríctamente decreciente), si $x < y$ implica que $f(x) > f(y)$ para todo $x,y \in A$.

\end{itemize}

\end{definition}

\section{Ejemplos de funciones}

\begin{definition}[Función lineal] \index{Función!Ejemplos!Función lineal}

La \textbf{función lineal} es de la forma $f : \mathbb{R} \to \mathbb{R}$ con 

$$ f(x) = mx + b $$

con $m,b \in \mathbb{R}$.

Su gráfica representa una \textbf{recta} en el plano, a $m$ se lo llama \textbf{pendiente} de la recta.

Dos rectas de ecuaciones $y = m_1 x + b_1$ e $y = m_2 x + b_2$ se dicen \textbf{perpendiculares} (se cortan formando un ángulo de $90^{\circ}$) sólamente si 

$$\boxed{m_2 = \frac{-1}{m_1}}$$

\end{definition}

\begin{definition}[Función módulo] \index{Función!Ejemplos!Función módulo}

Recordamos que (ver \ref{funcion_modulo}) la \textbf{función valor absoluto} (o módulo) es $|\cdot| : \mathbb{R} \to \mathbb{R}$ con

$$|x| = \begin{cases} x & \textrm{ si } x \geq 0 \\ -x & \textrm{ si } x < 0 \end{cases}$$

$f(\mathbb{R}) = \mathbb{R}_0^+$
\end{definition}

\begin{definition}[Función signo] \index{Función!Ejemplos!Función signo}
La \textbf{función signo} es $sg : \mathbb{R} - \{0\} \to \mathbb{R}$ con

$$sg(x) = \begin{cases} 1 & \textrm{ si } x > 0 \\ -1 & \textrm{ si } x < 0 \end{cases}$$
\end{definition}

\begin{definition}[Función cuadrática] \index{Función!Ejemplos!Función cuadrática}
La \textbf{función cuadrática} es $f : \mathbb{R} \to \mathbb{R}$ con

$$f(x) = ax^2 + bx + c$$

con $a,b,c \in \mathbb{R}$, y $a \neq 0$.

Su gráfica es una parábola cuyo eje de simetría es paralelo al eje de ordenadas.  Dicho eje de simetría es $x = -b/2a$

Sus ceros pueden obtenerse utilizando la fórmula resolvente (ver \ref{formula_resolvente}).  Si $\triangle < 0$ la función no tiene ceros (reales).
\end{definition}

\begin{definition}[Función polinómica] \index{Función!Ejemplos!Función polinómica}
Una \textbf{función polinómica} es de la forma $f:\mathbb{R} \to \mathbb{R}$ con 

$$f(x) = a_n x^n + a_{n-1} x^{n-1} + \ldots + a_1 x + a_0 $$

con $a_i \in \mathbb{R}$ con $0 \leq i \leq n$.
\end{definition}

\begin{definition}[Función racional] \index{Función!Ejemplos!Función racional}
Una \textbf{función racional} es de la forma $f : \mathbb{R} - C_0(q(x)) \to \mathbb{R}$ con $f(x) = p(x)/q(x)$ sinedo $p(x)$ y $q(x)$ polinomios, y $C_0(q(x))$ el conjunto de ceros reales de $q(x)$.

Si $gr(q) = 1$ y $gr(p) \leq 1$ se denomina \textbf{función homográfica}, se expresa como

$$f(x) = \frac{ax+b}{cx+d}$$

con $c \neq 0$.

Tiene asíntota vertical $x = -d/c$, y asíntota horizontal $y = a/c$.

La imagen es $Im(f) = \mathbb{R} - \{ a/c \}$

Si $f(x) = kx$ se conoce como \textbf{variación directa}, con constante de proporcionalidad $k$.

Si $f(x) = \frac{k}{x}$ se conoce como \textbf{variación inversa}, con constante de proporcionalidad $k$.
\end{definition}

\section{Función compuesta}

\begin{definition} \index{Función!Compuesta}
Sea $f : A \to B$ y $g : C \to D$ con $B \subseteq C$.  Entonces podemos definir la \textbf{función compuesta} $g \circ f : A \to D$, donde 

$$(g \circ f)(x) = g(f(x))$$
\end{definition}

\section{Función inyectiva, sobreyectiva, biyectiva}

Una función $f : A \to B$ se dice \textbf{inyectiva} \index{Función!Inyectiva} sii $f(a) = f(b)$ implica que $a=b$.  Equivaléntemente si $a \neq b$ implica que $f(a) \neq f(b)$.

Una función $f : A \to B$ se dice \textbf{sobreyectiva} \index{Función!Sobreyectiva}, si $f(A) = B$.

Si una función $f : A \to B$ es inyectiva y sobreyectiva, decimos que es \textbf{biyectiva} \index{Función!Biyectiva}.  Esto equivale a decir que existe una \textbf{función inversa} \index{Función!Función inversa} $g : B \to A$ tal que $g(f(x)) = f(g(x)) = x$

\section{Funciones trascendentes}

\begin{definition}[Función exponencial] \index{Función!Ejemplos!Función exponencial}
Una \textbf{función exponencial} es una función de la forma $f : \mathbb{R} \to \mathbb{R}^+$ con $f(x) = a^x$ con $a \in (0,1) \cup (1, +\infty)$

\end{definition}

\begin{property}
Propiedades de la función exponencial

\begin{itemize}

\item No presenta ceros.

\item $ a^{x+y} = a^x a^y$ 

\item $a^{x-y} = a^x / a^y$

\item Si $a \in (0,1)$ la función es estríctamente decreciente, y si $a \in (1, +\infty)$  la función es estrictamente creciente.

\item La recta $y=0$ es la asíntota horizontal, y no tiene asíntota vertical.

\item Es biyectiva, es decir tiene inversa: la función logarítmica.

\end{itemize}
\end{property}

\begin{definition}[Función logarítmica] \index{Función!Ejemplos!Función logarítmica}

Llamamos \textbf{función logarítmica} de base $a \in (0,1) \cup (1, +\infty)$ a la inversa de la función exponencial de base $a$.  

Es decir $g : \mathbb{R}^+ \to \mathbb{R}$ con $g(x) = \log_a(x)$.

Es decir que se verifica que

$$ \log_a(a^x) = a^{\log_a(x)} = x $$
	
\end{definition}

\begin{property}

Propiedades de la función logarítmica

\begin{itemize}

\item Es una función biyectiva.

\item Tiene un único cero $x=1$.

\item La recta de ecuación $x=0$ es asíntota vertical, y no tiene asíntota horizontal.

\item $ \log_a(xy) = \log_a(x) + \log_a(y)$ (para todo $x,y > 0$)

\item $ \log_a(x/y) = \log_a(x) - \log_a(y)$ (para todo $x,y > 0$)

\item $ \log_a(x^y) = y \log_a(x)$ para todo $y \in \mathbb{R}$, $x > 0$

\item Si conocemos $\log_a(x)$ y queremos conocer $\log_b(x)$ con $b \neq a$, podemos aplicar la fórmula

$$ \log_b(x) = \frac{\log_a(x)}{\log_a(b)}$$

que demostramos a continuación

$ y = \log_b(x)$

$ b^y = x $

$ \log_a(b^y) = \log_a(x) $

$ y \log_a(b) = \log_a(x) $

$ y = \frac{\log_a(x)}{\log_a(b)} $

$ \log_b(x) = \frac{\log_a(x)}{\log_a(b)} $
\end{itemize}
\end{property}



\chapter{Trigonometría}

\begin{definition}[Definiciones básicas] \index{Trigonometría}

Llamamos \textbf{plano euclídeo} al conjunto

$$ \mathbb{R}^2 = \{ (x,y) : x, y \in \mathbb{R} \} $$

cuyos elementos llamamos puntos o vectores, con las siguientes operaciones.

Sean $(x_1,x_2)$ e $(y_1,y_2)$ dos vectores.

La \textbf{suma} es $(x_1+y_1, x_2+y_2)$.  

Sea $\lambda \in \mathbb{R}$, entonces el \textbf{producto de un vector por un escalar} es $\lambda(x_1,x_2) = (\lambda x_1, \lambda x_2)$

El producto interno estándard (o \textbf{producto escalar}) es

$$ (x_1,x_2) \cdot (y_1,y_2) = x_1 y_1 + x_2 y_2 $$

La \textbf{norma} de un vector $(x_1, x_2)$ se define como

$$ ||(x_1,x_2)|| = \sqrt{ (x_1, x_2) \cdot (x_1,x_2) } = \sqrt{x_1^2 + x_2^2} $$

La \textbf{distancia} entre dos puntos $(x_1,x_2)$ e $(y_1,y_2)$ se define como

$$ d((x_1,x_2), (y_1,y_2)) = ||(x_1,x_2) - (y_1,y_2) || = \sqrt{(x_1-y_1)^2 + (x_2-y_2)^2} $$

La \textbf{circunferencia unitaria} $C$ es el conjunto de puntos que dista 1 del origen de coordenadas, es decir son los $(x,y) \in \mathbb{R}^2$ tales que

$$ d((x,y),(0,0)) = 1 $$

$$ \sqrt{(x-0)^2 + (y-0)^2} = 1 $$

$$ x^2 + y^2 = 1 $$

Dado un punto $(x,y) \in C$, el mismo forma cierto \textbf{ángulo} $\phi$ con el eje $x$.  Dicho ángulo $\phi$ puede medirse en distintas unidades, como grados, o radianes.  

En este apunte vamos a asumir que utilizamos siempre \textbf{radianes}, que equivale a la longitud del segmento de circunferencia unitaria correspondiente.  En particular un giro completo equivale a $2\pi$.

\end{definition}

\begin{figure}[h]
\centering\includegraphics[scale=0.5]{images/01_precalculo/trigonometry.png}
\caption{Trigonometría}
\end{figure}

\begin{figure}[h]
\centering\includegraphics[scale=0.6]{images/01_precalculo/hipotenusa.png}
\caption{Hipotenusa}
\end{figure}

\begin{definition}
Definimos las \textbf{funciones trigonométricas} $\cos(\phi) = x$ y $\sin(\phi) = y$, llamadas coseno de $\phi$ y seno de $\phi$ respectivamente.

En la siguiente figura podemos ver la circunferencia unitaria y un ángulo $\theta$ en verde.  Vemos que queda formado un triángulo rectángulo en rojo.


En términos de un triángulo rectángulo, llamamos \textbf{catetos} a los lados del ángulo rectángulo, e hipotenusa al otro lado.  En términos del ángulo $\phi$ entre la hipotenusa y un cateto, lo llamamos \textbf{cateto adyacente}, y al otro cateto lo llamamos \textbf{cateto opuesto}.

Se verifica entonces

$$\cos(\theta) = \frac{\textrm{Adyacente}}{\textrm{Hipotenusa}}$$

$$\sin(\theta) = \frac{\textrm{Opuesto}}{\textrm{Hipotenusa}}$$

Definimos también las \textbf{funciones trigonométricas} tangente, secante, cosecante y cotangente a continuación, siempre y cuando no se anule el denominador.

$$ \tan(\theta) = \frac{\sin(\theta)}{\cos(\theta)}$$

$$ \sec(\theta) = \frac{1}{\cos(\theta)}$$

$$ \mathrm{cosec}(\theta) = \frac{1}{\sin(\theta)}$$

$$ \cot(\theta) = \frac{1}{\tan(\theta)}$$

\end{definition}

\begin{observation}[Identidad trigonométrica fundamental] \index{Trigonometría!Identidad trigonométrica fundamental}
Para cualquier $\phi \in \mathbb{R}$ se verifica la identidad trigonométrica fundamental:

$$ \cos^2(\phi) + \sin^2(\phi) = 1$$
\end{observation}

\begin{proof}
Dado $\phi \in \RR$, como $x = \cos(\phi)$, $y = \sin(\phi)$ y $(x,y)$ es un punto de la circunferencia unitaria $x^2 + y^2 = 1$, se verifica $\cos^2(\phi) + \sin^2(\phi) = 1$.
\end{proof}



\section{Tabla de ángulos y valores} \index{Trigonometría!Tabla de ángulos y valores}

Puede ser útil conocer los siguientes datos sobre algunos ángulos.

\begin{figure}[h]
\centering\includegraphics[scale=0.5]{images/01_precalculo/unit_circle_angles_color.png}
\caption{Ángulos en la circunferencia unitaria}
\end{figure}

\begin{center}
  \begin{tabular}{| c | c | c | c | }
    \hline
    Radianes & Grados & Vértice & Aproximado \\ \hline \hline
    $0$ & $0$ & $(1,0)$ & $(1,0)$ \\ 
    $\pi/6$ & $30$ & $(\sqrt{3}, 1)/2$ & $(0.866, 0.5)$ \\
    $\pi/4$ & $45$ & $(\sqrt{2}, \sqrt{2})/2$ & ($0.707, 0.707)$ \\
    $\pi/3$ & $60$ & $(1, \sqrt{3})/2$ & $(0.5, 0.866)$ \\
    $\pi/2$ & $90$ & $(0, 1)$ & $(0, 1)$ \\
    $2\pi/3$ & $120$ & $(-1, \sqrt{3})/2$ & $(-0.5, 0.866)$ \\
    $3\pi/4$ & $135$ & $(-\sqrt{2}, \sqrt{2})/2$ & $(-0.707, 0.707)$ \\
    $5\pi/6$ & $150$ & $(-\sqrt{3}, 1)/2$ & $(-0.866, 0.5)$ \\
    $\pi$ & $180$ & $(-1, 0)$ & $(-1, 0)$ \\
    $7\pi/6$ & $210$ & $(-\sqrt{3}, -1)/2$ & $(-0.866, -0.5)$ \\
    $5\pi/4$ & $225$ & $(-\sqrt{2}, -\sqrt{2})/2$ & $(-0.707, -0.707)$ \\
    $4\pi/3$ & $240$ & $(-1, -\sqrt{3})/2$ & $(-0.5, -0.866)$ \\
    $3\pi/2$ & $270$ & $(0, -1)$ & $(0, -1)$ \\
    $5\pi/3$ & $300$ & $(1, -\sqrt{3})/2$ & $(0.5, -0.866)$ \\
    $7\pi/4$ & $315$ & $(\sqrt{2}, -\sqrt{2})/2$ & $(0.707, -0.707)$ \\
    $11\pi/6$ & $330$ & $(\sqrt{3}, -1)/2$ & $(0.866, -0.5)$ \\
    $2\pi$ & $0$ & $(1,0)$ & $(1,0)$ \\ 

    \hline
  \end{tabular}
\end{center}


\section{Funciones trigonométricas}

\begin{definition}[Seno] \index{Trigonometría!Función $\sin(x)$ }
La \textbf{función seno} es $\sin : \mathbb{R} \to \mathbb{R}$ con 

$$ x \to \sin(x)$$

A continuación la \textbf{gráfica} de la función seno
\end{definition}

\begin{figure}[h]
\centering\includegraphics[scale=0.5]{images/01_precalculo/sin.png}
\caption{$\sin(x)$}
\end{figure}

\begin{property}
Algunas propiedades de $\sin(x)$

\begin{itemize}
	
\item Sus \textbf{ceros} son $C_0(f) = \{ k\pi, \textrm{ con } k \in \mathbb{Z} \}$
	
\item Es \textbf{impar}, es decir $\sin(-x) = -\sin(x)$
	
\end{itemize}
\end{property}

\begin{definition}[Coseno] \index{Trigonometría!Función $\cos(x)$ }
La \textbf{función coseno} es $\cos : \mathbb{R} \to \mathbb{R}$ con 

$$ x \to \cos(x)$$

A continuación la gráfica de la función coseno.
\end{definition}

\begin{figure}[h]
\centering\includegraphics[scale=0.5]{images/01_precalculo/cos.png}
\caption{$\cos(x)$}
\end{figure}

\begin{property}
Algunas propiedades de $\cos(x)$

\begin{itemize}

\item Para todo ángulo se verifica que $\cos(x) = \sin(x + \frac{\pi}{2})$.  Es decir la curva se \textbf{traslada} en $\pi/2$ hacia la izquierda de la gráfica del seno.

\item Sus \textbf{ceros} son 

$$C_0(f) = \{ \frac{\pi}{2} + k\pi, \textrm{ con } k \in \mathbb{Z} \}$$

(son los de la función seno trasladados).

\item Es \textbf{par}, es decir $\cos(x) = \cos(-x)$

\end{itemize}
\end{property}

\begin{property}
Algunas propiedades comunes de $\sin(x)$ y $\cos(x)$

\begin{itemize}
\item La \textbf{imagen} tanto de $\sin(x)$ como de $\cos(x)$ es el intervalo $[-1,1]$.
	
\item Son \textbf{periódicas} con período $2\pi$, es decir $f(x) = f(x + 2\pi)$.  
	
O sea que el gráfico obtenido en el intervalo $[0, 2\pi)$ se repite periódicamente.
	
\item \textbf{No son inyectivas}.
\end{itemize}
\end{property}

\begin{definition}[Tangente] \index{Trigonometría!Función $\tan(x)$ }
La \textbf{función tangente} es $\tan : \mathbb{R} - C_0(\cos(x)) \to \mathbb{R}$, con 

$$ x \to \frac{\sin(x)}{\cos(x)} $$

A continuación su gráfica 
\end{definition}

\begin{figure}[h]
\centering\includegraphics[scale=0.5]{images/01_precalculo/tan.png}
\caption{$\tan(x)$}
\end{figure}

\begin{property}
Algunas propiedades de $\tan(x)$

\begin{itemize}
\item Los \textbf{ceros} de $\tan(x)$ coinciden con los de $\sin(x)$
	
\item Es \textbf{impar}, es decir $\tan(-x) = -\tan(x)$
	
\item Es \textbf{periódica} con período $\pi$.  Es decir $\tan(x) = \tan(x+\pi)$.
	
\item Tiene \textbf{asíntotas verticales} $x = \frac{\pi}{2} + k \pi$ con $k \in \mathbb{Z}$
\end{itemize}
\end{property}

\subsection{Las funciones $\sec(x)$ y $\mathrm{cosec}(x)$ y $\cot(x)$}

\begin{definition}[Secante, cosecante y cotangente] 
La \textbf{función secante} \index{Trigonometría!Función Secante} $\sec : \mathbb{R} - C_0(\sin(x)) \to \mathbb{R}$ con $x \to \frac{1}{\cos(x)} $ 

A continuación su gráfica

La \textbf{función cosecante} \index{Trigonometría!Función Cosecante} $\mathrm{cosec}(x) = \frac{1}{\sin(x)}$ es análoga a la función secante.

La \textbf{función cotangente} \index{Trigonometría!Función Cotangente} es $\cot(x) = \frac{1}{\tan(x)}$.  A continuación su gráfica
\end{definition}

\begin{figure}[h]
\centering\includegraphics[scale=0.5]{images/01_precalculo/sec.png}
\caption{$\sec(x)$}
\end{figure}

\begin{figure}[h]
\centering\includegraphics[scale=0.5]{images/01_precalculo/cot.png}
\caption{$\cot(x)$}
\end{figure}


\section{Funciones trigonométricas inversas}

Vimos que las funciones $\sin(x)$, $\cos(x)$ y $\tan(x)$ no son inyectivas.  

Restringiendo el dominio y el codominio podemos obtener una función biyectiva.

\begin{definition}
\begin{itemize}
Definimos las funciones trigonométricas inversas como sigue

\item Restringiendo $\sin(x)$ a $[-\frac{\pi}{2}, \frac{\pi}{2}] \to [-1,1]$ obtenemos una función biyectiva cuya inversa llamamos $ \arcsin(x) $. \index{Trigonometría!Función Arco-seno}

\item Restringiendo $\cos(x)$ a $[0, \pi] \to [-1,1]$ obtenemos una función biyectiva cuya inversa llamamos $ \arccos(x) $. \index{Trigonometría!Función Arco-coseno}

\item Restringiendo $\tan(x)$ a $(-\frac{\pi}{2}, \frac{\pi}{2}) \to \mathbb{R}$ obtenemos una función biyectiva cuya inversa llamamos $\arctan(x)$ \index{Trigonometría!Función Arco-tangente}

\end{itemize}
\end{definition}

\subsection{Ángulo respecto al eje $x$}

En el plano coordenado $\mathbb{R}^2$, es estandard medir ángulos respecto al lado positivo del eje $x$ en sentido antihorario, como muestra la figura

\begin{figure}[h]
\centering\includegraphics[scale=0.5]{images/01_precalculo/unit_circle_angles_color.png}
\caption{Ángulos en la circunferencia unitaria}
\end{figure}

Supongamos que queremos averiguar el ángulo que forma el vector $v = (-1,-1)$ respecto al eje $x$.  

Como $||v|| = \sqrt{2}$, sabemos que $\cos(\phi) = \sin(\phi) = - \sqrt{2}/2$

Ahora que pasa, la calculadora arroja los siguientes resultados: $\arccos(-\sqrt{2}/2) = 3 \pi/4 $ y $\arcsin(- \sqrt{2}/2) = - \pi / 4$.  

No sólo nos da dos resultados distintos entre sí, sino que ninguno de ellos es el buscado.  Notar que $3\pi/4$ pertenece al segundo cuadrante, $- \pi/4$ pertenece al cuarto cuadrante, y nuestro vector $(-1,-1)$ pertenece al tercer cuadrante.

La idea para obtener el ángulo correcto es analizar el cuadrante que queremos en el dibujo, y considerar ángulos semejantes en el cuadrante en cuestión.  En este caso queremos tercer cuadrante, y lo podemos obtener como $\pi + \pi/4 = 5\pi/4$

Ejercicio: Calcular el ángulo con respecto al eje $x$ de $(- \sqrt{3}, -1)$.  La respuesta es $7\pi/6 = 210^{\circ}$

\section{Función sinusoidal}

\begin{definition}[Función sinusoidal] \index{Función!Ejemplos!Función sinusoidal}
Una función $f$ se dice que es una \textbf{función sinusoidal} si es de la forma $f : \mathbb{R} \to \mathbb{R}$ con 

$$f(x) = K \sin(ax+b)$$

La curva gráfica de la función sinusoidal se llama onda sinusoidal, o sinusoide.  En el gráfico a continuación se representa una sinusoide.

\begin{itemize}
\item La \textbf{amplitud} es $\boxed{ K } $.

\item La \textbf{frecuencia} (ordinaria) es $ \boxed{f = \frac{a}{2\pi} }$ (cíclos por unidad de tiempo).

\item El \textbf{período} es $ \boxed{ \frac{1}{f} }$, el intervalo en que vuelve a empezar

\item La frecuencia angular es $a$.
\item La fase es $b$.
\item El ángulo de fase es $\frac{b}{a}$.  La curva se traslada en esa cantidad.

Si es negativo se retrasa, es decir se corre a la derecha.  Si es positivo se adelanta, es decir se corre a la izquierda.

\end{itemize}

Observar que la función coseno es sinusoidal porque $\cos(x) = \sin(x + \pi/2)$
\end{definition}

\begin{figure}[h]
\centering\includegraphics[scale=0.7]{images/01_precalculo/sinusoide.jpg}
\caption{Sinusoide}
\end{figure}

\section{Identidades trigonométricas}

\begin{definition}[Identidad trigonométrica] \index{Trigonometría!Identidad trigonométrica}
Una \textbf{identidad trigonométrica} es una ecuación que se verifican para todo valor del ángulo, por ejemplo la identidad trigonométrica fundamental 

$$ \cos^2(\phi) + \sin^2(\phi) = 1 $$
\end{definition}

\begin{definition}[Fórmula de Euler]
La \textbf{fórmula de Euler} establece que dado $\phi \in \mathbb{R}$, entonces

$$ e^{i \phi} = \cos(\phi) + i \sin(\phi) $$
\end{definition}

\begin{observation}
La exponencial compleja cumple la siguiente propiedad 

$$ e^{i (\alpha + \beta)} = e^{i \alpha} e^{i \beta} $$

Por lo tanto

\begin{eqnarray*} 
	e^{i (\alpha + \beta)} &=& \cos(\alpha + \beta) + i \sin(\alpha + \beta) \\
	&=& [\cos(\alpha) + i \sin(\alpha) ] [\cos(\beta) + i \sin(\beta)] \\
	&=& [\cos(\alpha)\cos(\beta) - \sin(\alpha)\sin(\beta)] + i [\cos(\alpha)\sin(\beta) + \sin(\alpha)\cos(\beta)]
\end{eqnarray*}
	
\end{observation}

\subsection{Suma de ángulos} 

Igualando parte real e imaginaria del número complejo obtenemos fórmulas para el seno y el coseno de la suma de dos ángulos:

\index{Trigonometría!Identidad trigonométrica!$\cos(\alpha + \beta)$ } \index{Trigonometría!Identidad trigonométrica!$\sin(\alpha + \beta)$ }

\begin{eqnarray*}
\cos(\alpha + \beta) &=& \cos(\alpha)\cos(\beta) - \sin(\alpha)\sin(\beta) \\
\sin(\alpha + \beta) &=& \cos(\alpha)\sin(\beta) + \sin(\alpha)\cos(\beta)
\end{eqnarray*}

También podemos verificar dichas identidades geométricamente mediante el siguiente gráfico 

\begin{figure}[h]
\centering\includegraphics[scale=0.7]{images/01_precalculo/cos_of_sum.png}
\caption{Coseno de la suma}
\end{figure}

Y recordando que $\cos(x)$ es par, y $\sin(x)$ es impar, obtenemos fórmulas para el seno y el coseno de la diferencia de dos ángulos:

\index{Trigonometría!Identidad trigonométrica!$\cos(\alpha - \beta)$ } \index{Trigonometría!Identidad trigonométrica!$\sin(\alpha - \beta)$ }

\begin{eqnarray*}
\cos(\alpha - \beta) &=& \cos(\alpha)\cos(\beta) + \sin(\alpha)\sin(\beta) \\
\sin(\alpha - \beta) &=& \sin(\alpha)\cos(\beta) -\cos(\alpha)\sin(\beta)
\end{eqnarray*}

Además, para la tangente de la suma de dos ángulos obtenemos

\index{Trigonometría!Identidad trigonométrica!$\tan(\alpha + \beta)$ } 

\begin{eqnarray*}
\tan(\alpha + \beta) &=& \frac{\sin(\alpha + \beta)}{\cos(\alpha + \beta)} \\
 &=& \frac{\cos(\alpha)\sin(\beta) + \sin(\alpha)\cos(\beta)}{\cos(\alpha)\cos(\beta) - \sin(\alpha)\sin(\beta)} \\
 &=& \frac{\frac{\cos(\alpha)\sin(\beta)}{\cos(\alpha)\cos(\beta)} + \frac{\sin(\alpha)\cos(\beta)}{\cos(\alpha)\cos(\beta)}}{\frac{\cos(\alpha)\cos(\beta)}{\cos(\alpha)\cos(\beta)} - \frac{\sin(\alpha)\sin(\beta)}{\cos(\alpha)\cos(\beta)}} \\
 &=& \frac{\tan(\beta) + \tan(\alpha)}{1 - \tan(\alpha)\tan(\beta)}
\end{eqnarray*}

\subsection{Suma de senos y de cosenos}

Se verifica entonces que

\begin{eqnarray*}
\cos(\alpha + \beta) + \cos(\alpha - \beta) &=& 2 \cos(\alpha)\cos(\beta) \\
\cos(\alpha + \beta) - \cos(\alpha - \beta) &=& -2 \sin(\alpha) \sin(\beta) \\
\sin(\alpha + \beta) + \sin(\alpha - \beta) &=& 2 \sin(\alpha) \cos(\beta) \\
\sin(\alpha + \beta) - \sin(\alpha - \beta) &=& 2 \cos(\alpha) \sin(\beta) 
\end{eqnarray*}

Si $a = \alpha + \beta$ y $b = \alpha - \beta$, se tiene que $\alpha = \frac{a+b}{2}$ y $\beta = \frac{a-b}{2}$, y 

\index{Trigonometría!Identidad trigonométrica!$\cos(a) + \cos(b)$} 
\index{Trigonometría!Identidad trigonométrica!$\cos(a) - \cos(b)$} 
\index{Trigonometría!Identidad trigonométrica!$\sin(a) + \sin(b)$} 
\index{Trigonometría!Identidad trigonométrica!$\sin(a) - \sin(b)$} 

\begin{eqnarray*}
\cos(a) + \cos(b) &=& 2 \cos(\frac{a+b}{2})\cos(\frac{a-b}{2}) \\
\cos(a) - \cos(b) &=& -2 \sin(\frac{a+b}{2})\sin(\frac{a-b}{2}) \\
\sin(a) + \sin(b) &=& 2 \sin(\frac{a+b}{2}) \cos(\frac{a-b}{2}) \\
\sin(a) - \sin(b) &=& 2 \cos(\frac{a+b}{2}) \sin(\frac{a-b}{2}) 
\end{eqnarray*}

\subsection{Algo extra...}

Usando la fórmula de Euler, y que coseno es par y seno es par

\begin{eqnarray*}
e^{i \alpha} &=& \cos(\alpha) + i \sin(\alpha) \\
e^{- i \alpha} &=& \cos(\alpha) - i \sin(\alpha)
\end{eqnarray*}

Sumando y restando respectivamente

\begin{eqnarray*}
e^{i \alpha} + e^{-i \alpha} &=& 2 \cos(\alpha) \\
e^{i \alpha} - e^{-i \alpha} &=& 2 i \sin(\alpha)
\end{eqnarray*}

Por lo tanto

\begin{eqnarray*}
\cos(\alpha) &=& \frac{e^{i \alpha} + e^{-i \alpha}}{2} \\
\sin(\alpha) &=& \frac{e^{i \alpha} - e^{-i \alpha}}{2i}
\end{eqnarray*}

\section{Pendiente de una recta} \index{Trigonometría!Pendiente de una recta}

\begin{observation}[Pendiente de una recta]
La \textbf{ecuación cartesiana} de una recta $R$ en el plano es de la forma

$$ y = mx + b$$

Decimos que $m$ es la \textbf{pendiente} de la recta, y que $b$ es la \textbf{ordenada al origen}.

Entonces la pendiente $m$ de la recta $R$ es igual a la tangente del ángulo $\phi$ entre el eje $x$ y la recta dada.
\end{observation}

\begin{figure}[h]
\centering\includegraphics[scale=1]{images/01_precalculo/pendiente_recta.png}
\caption{Pendiente de una recta}
\end{figure}

\begin{proof}
Supongamos que sólo conocemos dos puntos de la recta $(x_1, y_1)$ y $(x_2, y_2)$, con $x_1 \neq x_2$, como se muestra en la siguiente figura, y que queremos encontrar la ecuación cartesiana de la recta.

Se tiene que una \textbf{ecuación paramétrica} de la recta $R$ es

$$(x,y) = (x_1, y_1) + \lambda(x_2 - x_1, y_2 - y_1)$$

con $\lambda \in \mathbb{R}$.  O sea que

\begin{eqnarray*}
x &=& x_1 + \lambda (x_2 - x_1) \\
y &=& y_1 + \lambda (y_2 - y_1) 
\end{eqnarray*}

De la primera ecuación

$$ \lambda = \frac{x - x_1}{x_2 - x_1}$$

En la segunda, y llamando $\triangle x = x_2 - x_1$ y $\triangle y = y_2 - y_1$

\begin{eqnarray*}
y &=& y_1 + \frac{x - x_1}{x_2 - x_1} (y_2 - y_1) \\
y &=& \frac{\triangle y}{\triangle x} x + \left( y_1 - x_1 \frac{\triangle y}{\triangle x} \right)
\end{eqnarray*}

Finalmente encontramos la ecuación cartesiana de la recta

$$ y = mx + b $$

donde $m = \frac{\triangle y}{\triangle x}$ y $ b = \left( y_1 - x_1 \frac{\triangle y}{\triangle x} \right) $

Notar que también podemos escribir esta ecuación como

$$ y - y_1 = m (x - x_1) $$

Volviendo a ver el dibujo, observamos que podemos formar un triángulo rectángulo donde $\triangle x$ y $\triangle y$ son sus catetos.  Considerando el ángulo $\alpha$ entre la recta $y = y_1$ y la recta dada, vemos que $m = \frac{\triangle y}{ \triangle x} = \tan(\alpha)$, pues es el cociente del cateto opuesto con el cateto adyacente.

Es decir que la pendiente de la recta es igual a la tangente del ángulo $\alpha$ entre la recta $R$ y la recta horizontal $y = y_1$.  O lo que es lo mismo, entre la recta $R$ y el eje $x$ (es el mismo ángulo $\alpha$).

\end{proof}

\section{Teoremas del seno y del coseno}

\begin{theorem}[Teorema del Seno] \label{teorema_del_seno}

Sea $ABC$ un triángulo, llamamos $\alpha, \beta, \gamma$ a sus ángulos, y $a,b,c$ a sus lados opuestos.

Entonces,

$$ \boxed{ \frac{a}{\sin(\alpha)} = \frac{b}{\sin(\beta)} = \frac{c}{\sin(\gamma)} } $$

\end{theorem}

\begin{figure}[h]
\centering\includegraphics[scale=0.4]{images/01_precalculo/teorema_seno.png}
\caption{Teorema del seno}
\end{figure}

\begin{proof}
Si $h = b \sin(\alpha)$ es la altura desde el vértice $C$, entonces el área del triángulo es $A = \frac{ch}{2} = \frac{c b \sin(\alpha)}{2}$

Similarmente, trazando la altura desde los otros vértices obtenemos $A = \frac{ac \sin(\beta)}{2} = \frac{ab \sin(\gamma)}{2}$, es decir

$$ 2A = bc \sin(\alpha) = ac \sin(\beta) = ab \sin(\gamma) $$

dividiendo por $abc$

$$ \frac{2A}{abc} = \frac{\sin(\alpha)}{a} = \frac{\sin(\beta)}{b} = \frac{\sin(\gamma)}{c} $$

tomando el recíproco

$$ \frac{abc}{2A} = \frac{a}{\sin(\alpha)} = \frac{b}{\sin(\beta)} = \frac{c}{\sin(\gamma)} $$
\end{proof}

\begin{proof}
Ahora vamos a dar otra demostración, suponiendo que el triángulo es agudo.

Primero partimos el triángulo en dos triángulos rectángulos, trazando la altura $h$ desde el vértice $C$, como en la figura siguiente


Luego,

$$ h = b \sin(\alpha) = a \sin(\beta) $$

de donde

$$ \frac{a}{\sin(\alpha)} = \frac{b}{\sin(\beta)} $$

Si trazamos la altura desde el vértice $B$ obtenemos

$$ \frac{a}{\sin(\alpha)} = \frac{c}{\sin(\gamma)} $$

Finalmente

$$ \frac{a}{\sin(\alpha)} = \frac{b}{\sin(\beta)} = \frac{c}{\sin(\gamma)}$$

\end{proof}

\begin{theorem}[Teorema del Coseno] \label{teorema_del_coseno}

Sea $ABC$ un triángulo, llamamos $\alpha, \beta, \gamma$ a sus ángulos, y $a,b,c$ a sus lados opuestos.

Entonces,

$$ \boxed{a^2 = b^2 + c^2 - 2bc \cos(\alpha)} $$

y vale simétricamente para los otros lados, es decir

\begin{eqnarray*}
a^2 &=& b^2 + c^2 - 2bc \cos(\alpha) \\
b^2 &=& a^2 + c^2 - 2ac \cos(\beta) \\
c^2 &=& a^2 + b^2 - 2ab \cos(\gamma)
\end{eqnarray*}
\end{theorem}

\begin{figure}[h]
\centering\includegraphics[scale=0.4]{images/01_precalculo/teorema_seno.png}
\caption{Teorema del coseno}
\end{figure}

Observación: En el caso particular de que el triángulo sea rectángulo nos queda el teorema de Pitágoras.  Es decir este teorema es una generalización del teorema de Pitágoras.

\begin{proof}  Demostramos utilizando producto interno.

Situamos el vértice de ángulo $\alpha$ en el origen del plano euclídeo.  Los lados $b$ y $c$ los interpretamos como vectores de $\mathbb{R}^2$, y al lado $a$ como el vector $c-b$.  Luego

$ ||a||^2 = || c-b ||^2 = (c-b) \cdot (c-b) = c^2 - 2bc + b^2 = ||c||^2 + ||b||^2 - 2 ||b|| ||c|| \cos(\alpha) $

Es decir que

$$a^2 = b^2 + c^2 - 2 bc \cos(\alpha)$$

\end{proof}

\begin{proof}  Demostramos utilizando distancia euclídea.
	
Dispongamos sus vértices en el plano cartesiano de la siguiente manera

$A = (0,0)$, $B = (b, 0)$ y $C = (a \cos(\alpha), a \sin(\alpha))$

La distancia entre $B$ y $C$ es 

$$ d(B,C) = c = \sqrt{(a \cos(\alpha) - b)^2 + (a \sin(\alpha) - 0)^2 } $$

Por lo tanto

$$ c^2 = a^2 \cos^2(\alpha) - 2ab \cos(\alpha) + b^2 + a^2 \sin^2(\alpha)$$

$$ c^2 = a^2 + b^2 - 2ab \cos(\alpha) $$

Rotando la forma en la que ubicamos los vértices del triángulo sobre el plano, obtenemos las otras identidades.

\end{proof}

\begin{proof}
Ahora vamos a dar otra demostración, suponiendo que el triángulo es agudo.  Como antes trazamos la altura desde el vértice $C$.  Llamamos $H$ al punto del lado $AB$ donde que se intersecta con dicha altura.

Luego $AH = b \cos(\alpha) $ y $HB = c - AH = c - b \cos(\alpha)$

Por Pitágoras en el triángulo rectángulo $AHC$

$$ b^2 = AH^2 + h^2 $$

y por Pitágoras en el triángulo rectángulo $CHB$

$$ a^2 = h^2 + HB^2 $$

restando

$$b^2 - a^2 = AH^2 - HB^2 $$

reemplazando $HB$ y $AH$

$$b^2 - a^2 = b^2 \cos^2(\alpha) - (c - b \cos(\alpha))^2 $$

$$b^2 - a^2 = b^2 \cos^2(\alpha) - (c^2 - 2bc \cos(\alpha) + b^2 \cos^2(\alpha)) $$

$$b^2 - a^2 = b^2 \cos^2(\alpha) - (c^2 - 2bc \cos(\alpha) + b^2 \cos^2(\alpha)) $$

$$b^2 - a^2 = - c^2 + 2bc \cos(\alpha) $$

$$ a^2 = b^2 + c^2 - 2bc \cos(\alpha) $$
\end{proof}



\chapter{Física}

\begin{definition}
Llamamos \textbf{Espacio Euclídeo} al conjunto $\mathbb{R}^n$, a sus elementos los llamamos \textbf{vectores} \index{Física!Vectores}, y llamamos \textbf{escalares} a los elementos de $\mathbb{R}$.

Están definidas las operaciones suma de vectores, y multiplicación de un vector por un escalar, como definimos a continuación.

Sea $u,v \in \mathbb{R}^n$ y $k \in \mathbb{R}$, digamos $u = (u_1, u_2, \ldots, u_n)$ y $v = (v_1, v_2, \ldots, v_n)$.

Entonces la \textbf{suma de vectores} \index{Física!Vectores!Suma} la definimos como

$$ u + v = (u_1 + v_1, u_2 + v_2, \ldots, u_n + v_n) $$

y definimos el \textbf{producto de un vector por un escalar} \index{Física!Vectores!Producto con un escalar} como

$$ ku = (k u_1, k u_2, \ldots, k u_n) $$

También está definido el \textbf{producto escalar} \index{Física!Vectores!Producto escalar} (o producto interno estándard) entre dos vectores $u$ y $v$ como

$$ u \cdot v = \sum_{i=1}^n u_i v_i = u_1 v_1 + u_2 v_2 + \ldots + u_n v_n $$

Dos vectores $u$ y $v$ se dicen \textbf{ortogonales} \index{Física!Vectores!Ortogonales} sii $u \cdot v = 0$

Definimos la \textbf{norma} \index{Física!Vectores!Norma} del vector $u$ como

$$ |u| = \sqrt{u \cdot u} = \sqrt{u_1^2 + u_2^2 + \ldots + u_n^2}$$

Si $|u| = 1$ decimos que $u$ es un \textbf{versor} \index{Física!Vectores!Versor}.  Dado $v \in \mathbb{R}^n$, $v \neq 0$, llamamos \textbf{versor asociado} a $v$ al versor $v / |v|$

El \textbf{ángulo entre dos vectores} \index{Física!Vectores!Ángulo} $u$ y $v$ (no nulos) lo definimos como $ 0 \leq \phi \leq \pi $ tal que

$$ \cos(\phi) = \frac{u \cdot v}{||u|| ||v||}$$

Esto en realidad se deduce del teorema del coseno (Ver \ref{teorema_del_coseno}) de la siguiente manera.

Sea $ABC$ un triángulo, llamamos $\alpha, \beta, \gamma$ a sus ángulos, y $a,b,c$ a sus lados opuestos.

Llamamos también $u = \vec{AC}$ e $v = \vec{AB}$

Entonces, por el teorema del coseno

$$ \boxed{a^2 = b^2 + c^2 - 2bc \cos(\alpha)} $$

Por un lado $a^2 = || u-v ||^2 = \sum_{i=1}^n (u_i - v_i)^2 = \sum_i u_i^2 - 2 \sum_i u_i v_i + \sum_i v_i^2$

Por otro lado $b^2 + c^2 - 2bc \cos(\alpha) = \sum_i u_i^2 + \sum_i v_i^2 - 2 \sqrt{ \sum_i u_i^2 } \sqrt{ \sum_i v_i^2 } \cos(\phi)$

Igualando

$\sum_i u_i^2 - 2 \sum_i u_i v_i + \sum_i v_i^2 = \sum_i u_i^2 + \sum_i v_i^2 - 2 \sqrt{ \sum_i u_i^2 } \sqrt{ \sum_i v_i^2 } \cos(\phi)$

Cancelando

$$ \sum_i u_i v_i = \sqrt{ \sum_i u_i^2 } \sqrt{ \sum_i v_i^2 } \cos(\phi)$$

O sea

$$ u \cdot v = ||u|| ||v|| \cos(\phi)$$

\end{definition}

\begin{figure}[h]
\centering\includegraphics[scale=0.4]{images/01_precalculo/teorema_seno.png}
\caption{Triángulo}
\end{figure}

\begin{observation}
Si $u$ y $v$ no nulos son ortogonales, entonces el ángulo entre ellos es $\pi/2 = 90^{\circ}$
\end{observation}

\begin{definition}[Proyección] \index{Física!Vectores!Proyección}
Si $A$ y $B$ son vectores, la \textbf{proyección} de $A$ sobre $B \neq 0$ es el vector que representamos en la siguiente figura

y se calcula como

$$ proy_B(A) = \frac{A \cdot B}{||B||} \frac{B}{||B||}$$

\end{definition}

\begin{figure}[h]
\centering\includegraphics[scale=0.5]{images/01_precalculo/proyeccion_vector.png}
\caption{Proyección de un vector}
\end{figure}

\begin{proof}
Como $ A \cdot B = ||A|| \cdot ||B|| \cos(\theta)$, la norma de la proyección es $||A|| \cos(\theta) = \frac{A \cdot B}{||B||} $, y multiplicando por el versor asociado a $B$, es decir $\frac{B}{||B||}$, obtenemos el vector buscado $proy_B(A) = \frac{A \cdot B}{||B||} \frac{B}{||B||}$.
\end{proof}

\section{Sistema de fuerzas} \index{Física!Sistema de fuerzas}

En física, una \textbf{fuerza} es una acción capaz de imponer una aceleración sobre un cuerpo material.  Se representa mediante un vector. 

Recordemos las tres leyes del movimiento de \textbf{Newton}.

\begin{itemize}
\item La \textbf{primera ley de Newton} establece que todo cuerpo permanece en reposo o movimiento uniforme en linea recta, a menos que una fuerza actúe sobre él.

\item La \textbf{segunda ley de Newton} establece que la aceleración de un cuerpo de masa $m$ es igual a la fuerza resultante que actúa sobre el, dividido su masa, es decir

$$ a = \frac{F}{m} $$

dicho de otra forma

$$F = m \cdot a$$

Su unidad de medida es el Newton, $1 \ N = 1 \ \frac{kg \cdot m}{s^2} $

\item La \textbf{tercera ley de Newton} establece que si un cuerpo $A$ ejerce una fuerza $F_{A/B}$ sobre $B$, entonces $B$ ejerce una fuerza de igual magnitud y sentido contrario $F_{B/A}$ sobre $A$, es decir

$$ F_{A/B} = - F_{B/A} $$

\end{itemize}

Llamamos \textbf{sistema de fuerzas} al conjunto de fuerzas $F_1, F_2, \ldots, F_n$ que actúan sobre un cuerpo.

La \textbf{fuerza resultante} \index{Física!Sistema de fuerzas!Resultante} es $F_r = \sum_{i=1}^n F_i $, y la \textbf{fuerza equilibrante} \index{Física!Sistema de fuerzas!Equilibrante} es $F_e = - F_r$

En el plano, una fuerza $F = (x,y)N$ también la podemos representar como $F = (n,\alpha)$ donde $n = ||F||$ y $\alpha$ es el ángulo del vector $F$ con el eje $x$.

\section{Cinemática} \index{Física!Cinemática}

La \textbf{cinemática} es la parte de la física que estudia el \textbf{movimiento} de los cuerpos, abstrayendose de las causas que lo producen.  Los problemas que resuelve la cinemática son sobre calcular posición, velocidad y aceleración de un cuerpo.

El cuerpo vamos a modelarlo matemáticamente como un punto en el espacio, como si fuera un \textbf{partícula} puntual.

Para determinar su posición primero debemos fijar un \textbf{sistema de referencia}.  Es decir fijar un sistema de coordenadas ortogonales en el plano o el espacio.

De aquí en adelante sólo trabajaremos con partículas que se mueven sobre un plano que representaremos por $\mathbb{R}^2$.

La posición de una partícula queda entonces determinada por el \textbf{vector posición} $r$.

La \textbf{trayectoria} \index{Física!Cinemática!Trayectoria} que sigue una partícula es la función que determina el vector posición en función del tiempo, es decir determina la posición de la partícula en cada instante.

$$ \vec{x}(t) = (x_x(t), x_y(t))$$

En la siguiente figura representamos una trayectoria en el plano

\begin{figure}[h]
\centering\includegraphics[scale=0.5]{images/01_precalculo/desplazamiento.png}
\caption{Desplazamiento}
\end{figure}

Supongamos que consideramos un intervalo de tiempo $\triangle t = t_f - t_i$.  La \textbf{posición inicial} es $ x_i = x(t_i)$ y la \textbf{posición final} es $x_f = x(t_f)$.  Entonces

\begin{itemize}
\item El \textbf{desplazamiento} es 

$$\triangle x = x(t_f) - x(t_i)$$  

\item La \textbf{distancia recorrida} es $||\triangle x||$

\item La \textbf{velocidad media} es $ v_m = \frac{\triangle x}{ \triangle t} = \frac{x(t_f) - x(t_i)}{\triangle t}$

\item La \textbf{velocidad instantánea} es 

$$ v(t) = \lim_{\triangle t \to 0} \frac{x(t + \triangle t) - x(t)}{\triangle t}$$

\item La \textbf{aceleración media} es $ a_m = \frac{\triangle v}{ \triangle t} = \frac{v(t_f) - v(t_i)}{\triangle t}$

\item La \textbf{aceleración instantánea} es 

$$ a(t) = \lim_{\triangle t \to 0} \frac{v(t+\triangle t) - v(t)}{\triangle t}$$

\end{itemize}

\subsection{Movimiento rectilíneo uniforme (MRU)}

La trayectoria se dice que está en \textbf{movimiento uniforme} también conocido como \textbf{movimiento rectilíneo uniforme} (MRU) \index{Física!Cinemática!MRU} cuando se puede expresar como

$$ x(t) = x_i + v (t - t_i) $$

donde $x_i \in \mathbb{R}^n$ es la posición inicial, es decir $x_i = x(t_i)$, y $v \in \mathbb{R}^n$ es la velocidad (constante).

En este caso, la partícula se mueve sobre una recta a velocidad constante.  La aceleración es nula en todo momento.

\subsection{Movimiento uniformemente variado (MPUV, MRUV)}

La trayectoria se dice que está en \textbf{movimiento uniformemente acelerado} también conocido como \textbf{movimiento uniformemente variado} (MUV) \index{Física!Cinemática!MUV} cuando se puede expresar como

$$ x(t) = x_i + v_i (t-t_i) + \frac{1}{2}a (t-t_i)^2 $$

Donde $x_i \in \mathbb{R}^n$ es la posición inicial, es decir $x_i = x(t_i)$, además $v_i \in \mathbb{R}^n$ es la velocidad inicial, es decir $v_i = v(t_i)$, y $a \in \mathbb{R}^n$ es la aceleración (constante).

La velocidad resulta entonces

$$ v(t) = v_i + a(t-t_i) $$

El MUV lo podemos clasificar según si es rectilíneo o no:

\begin{itemize}
\item Cuando la velocidad inicial $v_i$ es un vector paralelo a la aceleración $a$, la partícula se mueve sobre una recta, y se dice la trayectoria está en \textbf{movimiento rectilíneo uniformemente variado} (MRUV) \index{Física!Cinemática!MRUV}.  

Por ejemplo el tiro vertical (ver \ref{tiro_oblicuo}) es un MRUV.

\item En caso contrario, cuando el vector velocidad inicial $v_i$ no es paralelo al vector aceleración $a$, la partícula se mueve sobre una parábola, y se dice que la trayectoria está en \textbf{movimiento parabólico uniformemente variado} (MPUV) \index{Física!Cinemática!MPUV}.  

Por ejemplo el tiro oblicuo (ver \ref{tiro_oblicuo}) es un MPUV.
\end{itemize}

\subsection{Caída de cuerpos (Tiro oblicuo y vertical)}

La fuerza de gravedad atrae los cuerpos hacia la tierra con una aceleración con intensidad de 

$$ |g| = 9.8 \frac{m}{s^2}$$

Se denomina \textbf{proyectil} todo cuerpo que se mueve sólamente sujeto a una velocidad inicial, y a la fuerza que ejerce su propio peso por la acción de la gravedad, sin considerar otras fuerzas como la resistencia del aire.

Es decir que su trayectoria se puede expresar como

$$ x(t) = x_i + v_i (t-t_i) + \frac{1}{2} g (t-t_i)^2 $$

con $x_i = (x_i, y_i)$ la posición cuando $t= t_i$, $v_i = (v_x,v_y)$ la velocidad cuando $t=t_i$, y $g = (0, -9.8)$.

Si el ángulo entre $v_i$ y la horizontal es de $90^{\circ}$ decimos que es \textbf{tiro vertical} \label{tiro_vertical} \index{Física!Cinemática!Tiro vertical}.  Sinó decimos que es \textbf{tiro oblicuo} \label{tiro_oblicuo}. \index{Física!Cinemática!Tiro oblicuo}

\begin{example}
Si lanzamos un proyectil con $||v_i|| = 400 m/s$ y un ángulo de elevación de $30^{\circ}$.  Considerando $g = 10 m/s^2$.

Entonces la velocidad inicial es

\begin{eqnarray*}
v_i &=& 400 ( \cos(30^{\circ}), \sin(30^{\circ})) \\
 &=& 400 ( \frac{\sqrt{3}}{2}, \frac{1}{2}) \\
 &=& 200 ( \sqrt{3}, 1) \\
 &=& ( 200 \sqrt{3}, 200) 
\end{eqnarray*}

Digamos que la posición inicial es $x_i = (0,0)$.  Luego tenemos que la trayectoria es

$$ x(t) = (0,0) + ( 200 \sqrt{3}, 200) t + \frac{1}{2} (0, -10) t^2 $$

y su velocidad

$$ v(t) = (200\sqrt{3}, 200) + (0, -10) t $$

Luego, a los 5 segundos está en $x(5) = (1000 \sqrt{3}, 875)$

La velocidad a los 5 segundos es $v(5) = (200 \sqrt{3}, 150)$

Se encontrará a 1000 metros de altura cuando

$$ 200 t - 5 t^2 = 1000 $$

Es decir cuando $t_1 = 20 - 10\sqrt{2} \approx 5.85786 $ y tambien cuando $t_2 = 20 + 10\sqrt{2} \approx 34.1421$

La altura máxima alcanzada por el proyectil es cuando $v_y = 0$ es decir

$$ 200 - 10t = 0 $$

O sea cuando $t = 20$, y la altura alcanzada es $x_y(20) = 2000$, en ese instante tiene un velocidad de $v(20) = (200 \sqrt{3}, 0)$

El alcance máximo ocurre cuando $x_y(t) = 0$ es decir cuando

$$ 200 t - 5t^2 = 0 $$

O sea, cuando $t = 0$ está en el punto inicial, el otro es $t = 40$.  Dicho alcance es $x_x(40) = 8000 \sqrt{3}$.  La velocidad con la que llega es $v(40) = (200 \sqrt{3}, -200)$
\end{example}